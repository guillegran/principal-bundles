%        File: fibrados.tex
%     Created: jue dic 06 11:00  2018 C
% Last Change: jue dic 06 11:00  2018 C
%
%\documentclass[12pt,a4paper]{amsart}
\documentclass[12pt,a4paper]{article}
\usepackage[utf8]{inputenc}
\usepackage[spanish, es-noquoting]{babel}
\usepackage[left=2.5cm,right=2.5cm,top=2.5cm,bottom=2.5cm]{geometry}
\usepackage{amsmath}
\usepackage{amsfonts}
\usepackage{amssymb}
\usepackage{amsthm, mathtools}
\usepackage{tikz,tikz-cd}
\usetikzlibrary{arrows, babel}
\usepackage{url}
\urlstyle{rm}
\usepackage[colorlinks=true,linktocpage=true,pagebackref=true,linkcolor=blue,urlcolor=blue]{hyperref}
\usepackage{graphicx}
\usepackage[tocflat]{tocstyle}
\usepackage{tocbibind}

%Fuente Palatino:
%\usepackage[sc]{mathpazo}
%Fuente Times:
%\usepackage{newtxtext}
%\usepackage{newtxmath}
%Fuente Libertine:
\usepackage{libertine}
\usepackage[libertine]{newtxmath}

\newtheorem{thm}{Teorema}[section]
\newtheorem{prop}[thm]{Proposición}
\newtheorem{lema}{Lema}
\newtheorem{corol}[thm]{Corolario}
\theoremstyle{definition} \newtheorem{defn}[thm]{Definición}
\theoremstyle{definition} \newtheorem{ejemplo}[thm]{Ejemplo}
\theoremstyle{definition} \newtheorem{ejercicio}[thm]{Ejercicio}
\theoremstyle{remark} \newtheorem*{obs}{Observación}

\def\pr{\mathrm{pr}}
\def\CC{\mathbb{C}}
\def\gg{\mathfrak{g}}
\def\xx{\mathtt{x}}
\def\ad{\mathrm{ad}}
\def\ZZ{\mathbb{Z}}
\def\RR{\mathbb{R}}
\def\KK{\mathbb{K}}
\def\SF{\mathbb{S}}
\def\TT{\mathbb{T}}
\def\NN{\mathbb{N}}
\def\HH{\mathbb{H}}
\def\PP{\mathbb{P}}
\def\id{\mathrm{id}}
\def\im{\mathrm{im}\ }
\def\cc{\mathbf{c}}
\def\eps{\varepsilon}
\newcommand{\ve}[1]{\mathbf{#1}}

\DeclarePairedDelimiter\esc{\langle}{\rangle}
\DeclarePairedDelimiter\norm{\lVert}{\rVert}

		      \let\emph\relax
		      \DeclareTextFontCommand{\emph}{\it\bfseries}


		      \title{\textbf{Geometría de superficies topológicas} \\ Introducción a los fibrados principales}
\author{Guillermo Gallego Sánchez}
\date{}


\begin{document}
\maketitle
\begin{abstract}
  En este trabajo introducimos la teoría de \textit{fibrados}, centrándonos particularmente en los \textit{fibrados principales}. Definimos las ideas de \textit{curvatura} y \textit{conexión} en un fibrado principal. Finalmente, usamos todas las herramientas expuestas para demostrar la construcción de Weil de las clases características. Se añade también un breve apéndice donde se relaciona la teoría de los fibrados principales con ciertas ideas físicas.
\end{abstract}

\tableofcontents

\section{Preliminares}
\subsection*{Grupos de Lie}
\begin{defn}
  Un \emph{grupo de Lie} $G$ es
  \begin{itemize}
    \item un grupo, y
    \item una variedad diferenciable tal que la operación del grupo,
      \begin{align*}
	\cdot :G\times G&\longrightarrow G\\ 
	  (g,h) &\longmapsto gh, 
	\end{align*}
	y la aplicación que consiste en invertir,
	\begin{align*}
	   G&\longrightarrow G\\ 
	   g &\longmapsto g^{-1}, 
	  \end{align*}
	  son aplicaciones diferenciables.
  \end{itemize}

  Para cada punto $g\in G$ definimos los difeomorfismos
  \begin{align*}
    L_g :G&\longrightarrow G\\ 
      x &\longmapsto gx 
    \end{align*}
    y
    \begin{align*}
      R_g :G&\longrightarrow G\\ 
        x &\longmapsto xg 
      \end{align*}
      llamados \emph{multiplicación por la izquierda y por la derecha}, respectivamente.
\end{defn}
\begin{defn}
  Un campo vectorial $\xi\in \mathfrak{X}(G)$ se dice \emph{invariante por la izquierda} si $$(L_{g})_* \xi_{g'}=\xi_{gg'}$$ para cualesquiera $g,g'\in G$. El corchete de Lie de dos campos invariantes por la izquierda es invariante por la izquierda, de modo que los campos invariantes por la izquierda forman un álgebra de Lie, que se denomina el \emph{álgebra de Lie de $G$} y se denota por $\gg$. Un campo invariante por la izquierda está completamente determinado por su valor en el elemento neutro $1_G$, de modo que $\gg \cong T_{1_G}G$. 

  Se llama la \emph{forma de Maurer-Cartan} a la $1$-forma $\theta\in \Omega^1(G;\gg)$ en $G$ con valores en $\gg$ definida por
  \begin{equation*}
    \theta_g=(L_{g^{-1}})_*:T_gG \rightarrow T_{1_G}G\cong \gg.
  \end{equation*}
\end{defn}

La teoría básica de los grupos de Lie nos dice que todos los campos del álgebra de Lie de un grupo de Lie $G$ son completos, de forma que hay una correspondencia biunívoca entre los elementos de $\gg$ y los subgrupos uniparamétricos de $G$. Si tenemos un campo $\xi\in \gg$ y un elemento $g \in G$, éste genera una curva integral en $G$ con punto inicial $g$ que denotaremos por $\gamma^\xi_g(t)$ . Si fijamos $g=1_G$, podemos distinguir un elemento concreto $\gamma^\xi_{1_G}(1)$. Esto nos permite dar la siguiente definición:
\begin{defn}
  Se define la \emph{aplicación exponencial} de $G$ como la aplicación
  \begin{align*}
    \exp :\gg&\longrightarrow G\\ 
    \xi &\longmapsto \gamma^\xi_{1_G}(1).
    \end{align*}
\end{defn}
Así, el subgrupo uniparamétrico generado por un elemento $\xi \in \gg$ es exactamente $\exp(t\xi)$.

Para cada $g\in G$ tenemos una aplicación diferenciable $\mathrm{Ad}_g=L_g\circ R_g^{-1}:G\rightarrow G$, esto es, $\mathrm{Ad}_gh=ghg^{-1}$. Esta aplicación define una representación lineal del grupo en el álgebra de Lie 
\begin{align*}
  \ad :G&\longrightarrow \mathrm{Aut}(\gg)\\ 
  g &\longmapsto (\mathrm{Ad}_g)_*, 
  \end{align*}
conocida como la \emph{representación adjunta}. Explícitamente
\begin{equation*}
  \ad_g \xi = \left. \frac{d}{dt} \right|_{t=0} (g\exp(t\xi)g^{-1}).
\end{equation*} 
Nótese que $R_g^*\theta=\ad_{g^{-1}}\circ \theta$.

\subsection*{Distribuciones e integrabilidad}
\begin{defn}
  Sea $M$ una variedad diferenciable de dimensión $n$. Una \emph{distribución}, que denotamos por $D\subset TM$, es una elección en cada punto $x\in M$, diferenciable en $x$, de un subespacio vectorial de dimensión $k<n$, $D_x\subset T_xM$.

  Sea $D\subset TM$ una distribución. Una subvariedad $N\subset M$ se llama una \emph{variedad integral} de $D$ si $T_xN=D_x$ en cada punto $x\in N$. Si cada punto de $M$ está contenido en una variedad integral de $D$, entonces decimos que la distribución $D$ es \emph{integrable}.
\end{defn}

El resultado clave en la teoría de distribuciones es el siguiente:
\begin{thm}[Frobenius]
  Una distribución $D\subset TM$ es integrable si y sólo si dados dos campos $X,Y$ en un entorno abierto $U\subset TM$ tales que $X_x, Y_x \in D_x$ para todo $x\in U$, entonces también $[X,Y]_x \in D_x$ para todo $x\in U$.
\end{thm}
\section{Fibrados}
\begin{defn}
  Un \emph{fibrado} $(E,B,p,F)$ consta de:
  \begin{itemize}
    \item un espacio topológico $B$ llamado la \emph{base} del fibrado,
    \item un espacio topológico $E$ llamado el \emph{espacio total} del fibrado,
    \item un espacio topológico $F$ llamado la \emph{fibra} del fibrado,
    \item una aplicación continua $p:E\rightarrow B$ tal que para cada punto $x\in B$ existe un entorno abierto $U$ de $x$ y un homeomorfismo $\varphi_U:p^{-1}(U) \rightarrow U \times F$, llamado \emph{trivialización}, tal que el siguiente diagrama conmuta
      \begin{center}
	\begin{tikzcd}
	  p^{-1}(U) \arrow{rr}{\varphi_U}\arrow{rd}{p} && U \times F \arrow{ld}{\pr_1}	  \\ 
	  &U,&
	\end{tikzcd}
      \end{center}
	donde $\pr_1$ denota la proyección al primer factor $(x,y)\mapsto x$.
  \end{itemize}
  Generalmente denotaremos un fibrado por $E\rightarrow B$. Si no se especifica, $p$ denotará la aplicación y $F$ la fibra. Si $x\in B$ es un punto, llamamos a $F_x=p^{-1}(x)$ la \emph{fibra sobre $x$}.
\end{defn}
  \begin{ejemplo}
 El ejemplo más simple que podemos considerar es el \emph{fibrado trivial}, $(B\times F, B, \pr_1, F)$, cuyo espacio total es simplemente el producto cartesiano y la aplicación es la proyección al primer factor.
    \qed
  \end{ejemplo}
  \begin{ejemplo}\label{moebius}
    Un ejemplo de fibrado no trivial viene dado por la \emph{cinta de Moebius}. El espacio total es la susodicha cinta, que podemos describir por
    \begin{equation*}
      E=\frac{\left\{ (x,y): x\in \RR, y \in (0,1) \right\}}{(x,y)\sim (x+1,1-y)},
    \end{equation*}
    mientras que la base es la circunferencia $$B=\frac{ \left\{ x\in \RR \right\}}{x \sim x+1} \cong \SF^1.$$ La aplicación $p:E\rightarrow B$ es la proyección en la primera coordenada. 
    \qed
  \end{ejemplo}

    Consideremos ahora $E\rightarrow B$ un fibrado y $x\in B$ un punto. Sean $U$ y $V$ dos entornos abiertos de $x$ con trivializaciones $\varphi_U$ y $\varphi_V$ respectivamente. El siguiente diagrama conmutativo
    \begin{center}
      \begin{tikzcd}
	U\cap V \times F \arrow{r}{\varphi^{-1}_U} \arrow{rd}{\pr_1} & p^{-1}(U\cap V) \arrow{d}{p} \arrow{r}{\varphi_V}& U\cap V \times F \arrow{ld}{\pr_1} 	\\ 
&	U\cap V,&
      \end{tikzcd}
    \end{center}
      induce un homeomorfismo
      \begin{align*}
	\varphi_V \circ \varphi_U^{-1} :(U\cap V) \times F&\longrightarrow (U\cap V)\times F\\ 
	(x,y) &\longmapsto (x,\psi_{UV}(x,y)),
	\end{align*}
	para cierta $\psi_{UV}:U\cap V \times F \rightarrow F$.
	En particular, la aplicación
	\begin{align*}
	  \psi_{UV,x} :F&\longrightarrow F\\ 
	  y &\longmapsto \psi_{UV}(x,y), 
	  \end{align*}
	  es un homeomorfismo. Se llama \emph{función de transición} entre $U$ y $V$ a la aplicación
	  \begin{align*}
	    g_{UV} :U\cap V&\longrightarrow \mathrm{Homeo}(F)\\ 
	    x &\longmapsto \psi_{UV,x}. 
	    \end{align*}
	    Nótese que, si $U,V,W$ son abiertos trivializantes, entonces las funciones de transición cumplen la \emph{condición de cociclo}
	    \begin{equation*}
	      g_{UW}=g_{VW}\circ g_{UV}.
	    \end{equation*}
	    En particular, $g_{UU}=\id$ y $g_{UV}=g_{VU}^{-1}$.

	    Las funciones de transición nos permiten distinguir distintos tipos de fibrados. Es decir, si los elementos que constituyen el fibrado pertenecen a cierta categoría geométrica (por ejemplo, la de las variedades diferenciables) y las funciones de transición son isomorfismos en esa categoría (por ejemplo, difeomorfismos), el fibrado será de un tipo especial, relacionado con esa categoría (por ejemplo, un fibrado diferenciable). Un caso particularmente interesante es el siguiente:
	    \begin{defn}
	      Un \emph{fibrado vectorial} $E\rightarrow B$ es un fibrado cuya fibra es un $k$-espacio vectorial $V$ y tal que sus funciones de transición son de la forma 
	      \begin{align*}
		g_{UW}:U\cap W&\longrightarrow \mathrm{Aut}(V), 
		\end{align*}
		donde $\mathrm{Aut}(V)$ denota el conjunto de automorfismos de $V$. En particular, si $V$ es de dimensión finita $n$, entonces las funciones de transición son de la forma
	      \begin{align*}
		g_{UW}:U\cap W&\longrightarrow \mathrm{GL}(n,k), 
		\end{align*}
		y al evaluarlas en un punto $g_{UW}(x)$ se llaman \emph{matrices de transición}.
	    \end{defn}

	    Lo que recalca aún más la importancia de las funciones de transición es que se puede recuperar el fibrado a partir de éstas. En efecto, si $B$ y $F$ son espacios topológicos, $G<\mathrm{Homeo}(F)$ es un subgrupo del grupo de homeomorfismos de $F$ y $\mathcal{U}$ es un recubrimiento abierto de $B$, decimos que un conjunto de funciones de la forma
	    \begin{equation*}
	      \left\{ g_{UV}:U\cap V\rightarrow G:U,V \in \mathcal{U} \right\}
	    \end{equation*}
	    es un \emph{$1$-cociclo de \v{C}ech subordinado a $\mathcal{U}$ con coeficientes en $G$} si cumple la condición de cociclo
	    \begin{equation*}
	      g_{UW}=g_{VW}\circ g_{UV},
	    \end{equation*}
	    para $U,V,W\in \mathcal{U}$. A partir de un $1$-cociclo de \v{C}ech podemos construir un fibrado $E\rightarrow B$ cuyas funciones de transición en los abiertos de $\mathcal{U}$ son precisamente las del cociclo, de la siguiente manera:
	    \begin{itemize}
	      \item El espacio total es 
		\begin{equation*}
		  E=\bigsqcup_{U\in \mathcal{U}}(U\times F)/\sim,
		\end{equation*}
		con la relación de equivalencia $(x,y)\sim (x,g_{UV}(x)(y))$, con $x\in U\cap V$. 
	      \item La aplicación $p:E\rightarrow B$ es simplemente la proyección al primer factor $p:[(x,y)]\mapsto x$. 
	    \end{itemize}

	    Cabe preguntarse ahora cuándo dos $1$-cociclos de \v{C}ech en un recubrimiento $\mathcal{U}$ generarán el mismo fibrado. Para ello, es necesario concretar qué entendemos por «ser el mismo fibrado»:
	    \begin{defn}
	      Sean $p:E\rightarrow B$ y $p':E'\rightarrow B'$ dos fibrados con fibra $F$. Un \emph{isomorfismo de fibrados} entre $p:E\rightarrow B$ y $p':E'\rightarrow B$ es un par de homeomorfismos $(f,\tilde{f})$, $f:B'\rightarrow B$ y $\tilde{f}:E'\rightarrow E$ tal que el siguiente diagrama conmuta
	      \begin{center}
		\begin{tikzcd}
		  E'\arrow{r}{\tilde{f}}\arrow{d}{p'} & E\arrow{d}{p}		  \\ 
		  B' \arrow{r}{f} & B.
		\end{tikzcd}
	      \end{center}
	    \end{defn}

	    Si consideramos dos fibrados $p:E\rightarrow B$ y $p':E'\rightarrow B'$, un isomorfismo $(f,\tilde{f})$ entre ellos y tomamos dos trivializaciones $\varphi_{U'}:p'^{-1}(U')\rightarrow U'\times F$ y $\varphi_U:p^{-1}(U)\rightarrow U\times F$, tenemos una aplicación
	    \begin{align*}
	      \tilde{f}_{UU'}= \varphi_U\circ \tilde{f} \circ \varphi_{U'}^{-1} :U'\times F&\longrightarrow U\times F\\ 
	      (x',y') &\longmapsto (f(x'),\hat{f}_{UU',x}(y')), 
	      \end{align*}
	      con $\hat{f}_{UU',x'}:F\rightarrow F$ un homeomorfismo. Tenemos entonces una aplicación 
	      \begin{align*}
		f_{UU'} :U'&\longrightarrow \mathrm{Homeo}(F)\\ 
		x' &\longmapsto \tilde{f}_{UU',x'}. 
		\end{align*}
		Ahora, si consideramos otras dos trivializaciones $\varphi_{V'}:p'^{-1}(V')\rightarrow V'\times F$ y $\varphi_V:p^{-1}(V)\rightarrow V\times F$, entonces $\tilde{f}_{VV'}=\varphi_V\circ \tilde{f} \circ \varphi_V^{-1}$, de modo que tenemos la relación
		\begin{equation*}
		  \tilde{f}_{VV'}=\varphi_V\circ \varphi_U^{-1} \tilde{f}_{UU'}\varphi_{U'}\varphi_{V'}^{-1}.
		\end{equation*}
		Por tanto, 
		\begin{equation*}
		  f_{VV'}=g_{UV}\circ f_{UU'}\circ g'_{V'U'},
		\end{equation*}
		de donde
		\begin{equation*}
		  g'_{U'V'}=f_{VV'}^{-1}\circ g_{UV}\circ f_{UU'}.
		\end{equation*}
		Esta última igualdad nos permite establecer una relación de equivalencia en los $1$-cociclos de un recubrimiento $\mathcal{U}$ con coeficientes en $G<\mathrm{Homeo}(F)$. El cociente que resulta de esta relación de equivalencia se llama \emph{primer grupo de cohomología de \v{C}ech subordinado a $\mathcal{U}$ con coeficientes en $G$} y se denota por $\check{H}^1(\mathcal{U},G)$. Se puede ver que el refinamiento del recubrimiento $\mathcal{U}$ induce homomorfismos en los grupos de cohomología de \v{C}ech, de modo que se puede considerar el límite directo
		\begin{equation*}
		  \check{H}^1(B,G)=\varinjlim_{\mathcal{U}}\check{H}^1(\mathcal{U},G).
		\end{equation*}
		Este grupo se denomina \emph{primer grupo de cohomología de \v{C}ech con coeficientes en $G$} y, como hemos visto, clasifica los fibrados con fibra $F$.
	    

	    \begin{defn}
	      Sea $E\rightarrow B$ un fibrado. Una \emph{sección} del fibrado es una aplicación continua $s:B\rightarrow E$ tal que $p\circ s = \id_B$. Denotamos $\Gamma(E)$ al conjunto de las secciones del fibrado $E\rightarrow B$ y $\Gamma(U,E)$ a las secciones locales definidas en un abierto $U\subset E$.
	    \end{defn}
	    \begin{obs}
	      En el caso del fibrado trivial, $\Gamma(U,E)=C(U,F)$, es decir, las secciones son las aplicaciones continuas del abierto $U$ a la fibra.
	    \end{obs}

	    Sea $s$ una sección y $U_1$ y $U_2$ dos abiertos trivializantes. Si consideramos las secciones locales $s_{i}:U_i \rightarrow F$ tales que $\varphi_i \circ s(x)=(x,s_i(x))$, entonces 
	    \begin{equation*}
	      s_j = g_{ij} s_i,
	    \end{equation*}
	    con $g_{ij}=g_{U_iU_j}$ las funciones de transición. Recíprocamente, si defino secciones locales $s_U$ en cada abierto $U\in \mathcal{U}$ de un recubrimiento $\mathcal{U}$ de $B$ por abiertos trivializantes, puedo recuperar una sección $s\in \Gamma(E)$.
	      
	    \section{Fibrados principales. Definición y ejemplos}
	    \begin{defn}
	      Un \emph{fibrado principal} $(P,B,p,G)$ consta de:
		  \begin{itemize}
		    \item una variedad diferenciable $P$,
		    \item un grupo de Lie $G$ actuando libremente por la derecha sobre $P$:
		      \begin{align*}
			 P\times G&\longrightarrow P\\ 
			  (p,g) &\longmapsto p\cdot g ,
			\end{align*}
		    \item $B=P/G$, con una sumersión $p:P\rightarrow P/G$ que es la proyección canónica al cociente.
		  \end{itemize}
		  Además, se cumple la condición de trivialidad local: para cada $x \in B$ existe un entorno $U$ de $x$ y un difeomorfismo $\varphi_U:p^{-1}(U) \rightarrow U \times G$ tal que el siguiente diagrama conmuta
      \begin{center}
	\begin{tikzcd}
	  p^{-1}(U) \arrow{rr}{\varphi_U}\arrow{rd}{p} && U \times G \arrow{ld}{\pr_1}	  \\ 
	  &U,&
	\end{tikzcd}
      \end{center}
      y tal que $\varphi_U(y)=(p(y),g_U(y))$ para cierta aplicación $g_U:p^{-1}(U)\rightarrow G$ con $g_U(y\cdot g)=g_U(y)\cdot g$.
	    \end{defn}
      En resumen, podemos pensar en un fibrado principal simplemente como un fibrado sobre una variedad diferenciable cuya fibra es un grupo de Lie.
	    Si $P\rightarrow B$ es un fibrado principal y $x\in B$, podemos considerar una trivialización $\varphi_U:p^{-1}(U)\rightarrow U \times G$ y una sección local $s_U:U\rightarrow P$ tal que $\varphi_U\circ s_U(x)=(x,1_G)$. Recíprocamente, para cada sección local $s_U:U\rightarrow P$ y para cada $y \in F_x$ hay un único elemento $g_U(y)\in G$ tal que $y=s_U(x) g_U(y)$, de modo que podemos identificar la fibra $F_x$ con $G$. Por otra parte, las funciones de transición son de la forma
	    \begin{align*}
	      \varphi_V\circ \varphi_U^{-1} :U\cap V \times G&\longrightarrow U\cap V \times G\\ 
	      (x,h) &\longmapsto (x,g_U(x,h) g_V(x,h)^{-1}h). 
	      \end{align*}
	      De hecho, si llamamos $\bar{g}_{UV}(y)=g_{U}(y)g_V(y)^{-1}$, tenemos que el valor de $\bar{g}_{UV}$ no varía en la fibra en la que se encuentra $y$:
	      \begin{equation*}
		\bar{g}_{UV}(y\cdot g)=g_{U}(y\cdot g)g_V(y\cdot g)^{-1}=g_{U}(y)g g^{-1}g_V(y)^{-1}=g_{U}(y)g_V(y)^{-1}=\bar{g}_{UV}(y),
	      \end{equation*}
	      de modo que $\bar{g}_{UV}(y)=g_{UV}(p(y))$ para cierta función
	      \begin{align*}
		g_{UV} :U\cap V&\longrightarrow G.
		\end{align*}
		Por tanto, las funciones de transición son de la forma
		\begin{align*}
		   U\cap V \times G&\longrightarrow U\cap V \times G\\ 
		   (x,h) &\longmapsto (x,g_{UV}(x)h). 
		  \end{align*}

		  Un primer resultado importante sobre fibrados principales concierne a sus secciones:
		  \begin{prop}
		    Un fibrado principal admite una sección global si y sólo si es trivial.
		  \end{prop}
		  \begin{proof}
		    En una dirección está claro, si $\pr_1:B\times G\rightarrow B$ es un fibrado trivial, cualquier función diferenciable $B\rightarrow G$, por ejemplo, la que manda todos los puntos al elemento neutro, define una sección global.
		    Por otra parte, sea $s:B\rightarrow P$ una sección de $P\rightarrow B$. Definimos la aplicación
		    \begin{align*}
		      \varphi: B\times G&\longrightarrow P\\ 
		        (x,g) &\longmapsto (s(x)\cdot g),
		      \end{align*}
		      que es diferenciable por ser composición de aplicaciones diferenciables. Veamos que es biyectiva. En efecto es inyectiva ya que si $s(x)\cdot g=s(y)\cdot g'$, $$y=p(s(y))=p(s(y)\cdot g')=p(s(x) \cdot g)=p(s(x))=x,$$ luego $x=y$, mientras que $g=g'$ por ser la acción libre. Por otra parte, es sobreyectiva ya que si $y\in P$, $F_{p(y)}$ es la órbita de $y$ por la acción de $G$ y, como $s(p(y))\in F_{p(y)}$, existe un $g\in G$ tal que $s(p(y))\cdot g=y$. 
		  \end{proof}

		  \begin{ejemplo}\label{bormob}
		    Podemos considerar $P=B=\SF^1\subset \CC$ y 
		    \begin{align*}
		      p :\SF^1&\longrightarrow \SF^1\\ 
		        z &\longmapsto z^2.
		      \end{align*}
		      Esto da un fibrado principal con fibra $\ZZ_2$, por medio de la acción
		      \begin{align*}
			 \SF^1 \times \ZZ_2&\longrightarrow \SF^1\\ 
			  (z,\pm 1) &\longmapsto \pm z. 
			\end{align*}
			Visualmente puede verse como la proyección del borde de la cinta de Moebius a una circunferencia interior. Se trata de un fibrado no trivial, ya que no admite una sección global. En efecto, una sección global de $p$ sería una raíz cuadrada univaluada en toda la circunferencia y sabemos que eso no puede existir.
			\qed
		  \end{ejemplo}
		  \begin{ejemplo}
		    Sea $M$ una variedad diferenciable conexa con recubridor universal $\tilde{M}$. La proyección recubridora $p:\tilde{M}\rightarrow M$ da un fibrado principal con fibra $G=\pi_1(M)$ el grupo fundamental de $M$ (en cualquier punto). La acción de $G$ en $\tilde{M}$ es simplemente la acción de monodromía: si $x\in M$ e $y\in F_x$, a cada elemento $g\in G$ le podemos asignar un representante $\gamma_g$ que es un lazo en $M$ con punto base $x$; este lazo levanta a un único camino $\tilde{\gamma}^y_g$ en $\tilde{M}$ con $\tilde{\gamma}^y_g(0)=y$ y podemos definir $y\cdot g=\tilde{\gamma}^y_g(1)$. Además, si $H\lhd G$ es un subgrupo normal de $G$ entonces $\tilde{M}/H\rightarrow M$ es un recubridor regular y también es un fibrado principal con fibra $G/H$.
		  \qed
		  \end{ejemplo}

		  \begin{ejemplo}
		    El ejemplo más importante que vamos a considerar en esta sección es el de \emph{fibrado de referencias}. Sea $M$ una variedad diferenciable, el fibrado de referencias está dado por el conjunto
		    \begin{equation*}
		      L(M)=\left\{ \mathcal{B}_x: x\in M,\  \mathcal{B}_x \text{ es una base de } T_xM \right\},
		    \end{equation*}
		     con la aplicación
		     \begin{align*}
		       p :L(M)&\longrightarrow M\\ 
		       \mathcal{B}_x &\longmapsto x. 
		       \end{align*}

		       Ahora, en coordenadas locales $(x^1,\dots,x^n)$ en un abierto $U\subset M$, los vectores de una base $\mathcal{B}_x=\left\{ X_{1,x},\dots,X_{n,x} \right\}$ se escribirán como
		       \begin{equation*}
			 X_{i,x}=\sum_{j=1}^n a_i^j \frac{\partial}{\partial x^j}.
		       \end{equation*}
		       Estos coeficientes $a_i^j$ pueden recogerse en una matriz $A$ con determinante no nulo y tenemos trivializaciones de la forma:
		       \begin{align*}
			 p^{-1}(U)&\longrightarrow U\times \mathrm{GL}(n,\RR)\\ 
			 \mathcal{B}_x &\longmapsto (x,A). 
			 \end{align*}
			 Las funciones de transición entre dos abiertos $U$ y $V$ vienen dadas por los cambios de coordenadas y son de la forma $g_{UV} \rightarrow \mathrm{GL}(n,\RR)$. Esto nos da una primera pista de la conexión entre este tipo de fibrados y los fibrados vectoriales.

			 Veamos que, en efecto, el fibrado de referencias es un fibrado principal con fibra $\mathrm{GL}(n,\RR)$. Si $A=(a^j_i)\in \mathrm{GL}(n,\RR)$ y $\mathcal{B}_x=\left\{ X_{1,x},\dots,X_{n,x} \right\}$ es una base de $T_xM$, podemos definir la acción simplemente como
			 \begin{equation*}
			   \mathcal{B}_x\cdot A =\left\{ X'_{1,x},\dots,X'_{n,x} \right\},
			 \end{equation*}
			 con $X'_{i,x}=\sum_{j=1}^n a^j_i X_{i,x}$. Equivalentemente, podemos ver los elementos de $L(M)$ como isomorfismos lineales $\psi:\RR^n \rightarrow T_{p(\psi)} M$ y la acción de $\mathrm{GL}(n,\RR)$ es simplemente
			 \begin{center}
			   \begin{tikzcd}
			     \RR^n \arrow{r}{A} &\RR^n \arrow{r}{\psi} &T_{p(\psi)}M.
			   \end{tikzcd}
			 \end{center}
			\qed   
		  \end{ejemplo}
      \section{Conexiones en fibrados principales}
      La idea de una conexión en geometría viene de un problema muy intuitivo, el del \emph{transporte paralelo}. Supongamos que tenemos un fibrado $E\rightarrow B$ y una curva $\gamma:[0,1]\rightarrow B$. La necesidad de un transporte paralelo se ve evidente a la hora de «derivar» secciones. Es decir, dada una sección $s:B\rightarrow E$, queremos definir su derivada a lo largo de la curva $\gamma$. Para hacer esto necesitamos dar una noción de cociente incremental
      \begin{equation*}
	\Delta s= \frac{s(\gamma(t+h))-s(\gamma(t))}{h}.
      \end{equation*}
      Sin embargo, si $\gamma(t+h)$ y $\gamma(t)$ son puntos distintos, sus secciones estarán en fibras distintas y la fórmula anterior es absurda, pues pretendemos restar dos vectores que se encuentran en espacios vectoriales distintos. Es por ello que queremos «conectar» las distintas fibras, de modo que la operación anterior tenga sentido. Una forma de conectar las fibras es considerar el fibrado tangente del espacio total. Si tomamos un punto $x\in B$ y un punto $y\in F_x$, entre los vectores tangentes a $E$ en $y$ habrá algunos que sean también tangentes a la fibra, que llamaremos «verticales». Los que nos permiten conectar la fibra $F_x$ con otras distintas son precisamente aquellos que no son verticales, y será en esos en los que centraremos nuestro interés. En un fibrado en general se obtiene la noción de \emph{conexión de Ehresmann}; nosotros sólo consideraremos conexiones en fibrados principales.

      Sea $p:P\rightarrow B$ un fibrado principal con fibra $G$. Sea $x\in B$ e $y\in F_x$. Se define el \emph{subespacio vertical} como $V_y=\ker p_* \subset T_yY$. Un campo vectorial $X$ en $P$ se dice \emph{vertical} si $X_y\in V_y$ para cada $y\in P$. El corchete de Lie de dos campos verticales es vertical. Además, si consideramos un elemento $g\in G$ y la acción 
      \begin{align*}
	 R_g:P&\longrightarrow P\\ 
	  y &\longmapsto y\cdot g, 
	\end{align*}
	entonces $R_{g,*}V_y=V_{y\cdot g}$. Decimos entonces que $V\subset TP$ es una \emph{distribución $G$-invariante}. Una conexión será simplemente otra distribución $G$-invariante complementaria a esta:
	\begin{defn}
	  Una \emph{conexión} en $P$ es una distribución de subespacios \emph{horizontales} $H_y\subset T_yY$ complementarios a $V_y$:
	  \begin{equation*}
	    T_yP=V_y \oplus H_y
	  \end{equation*}
	  y tal que $R_{g,*}H_y=H_{y\cdot g}$.
	\end{defn}

	La acción de $G$ en $P$ define una aplicación $\sigma:\gg\rightarrow \mathfrak{X}(P)$, que a cada $\xi\in \gg$ le asigna el \emph{campo fundamental} $\sigma(\xi)$ cuyo valor en un punto $y\in P$ es
	\begin{equation*}
	  \sigma_y(\xi)=\left.\frac{d}{dt}\right|_{t=0}(y\cdot \exp(t\xi)).
	\end{equation*}
	Nótese que 
	\begin{equation*}
	  p_* \sigma_y(\xi)=\left. \frac{d}{dt} \right|_{t=0}p(y\cdot \exp(t\xi))=\left. \frac{d}{dt} \right|_{t=0}p(y)=0,
	\end{equation*}
	de modo que $\sigma(\xi)$ es un campo vertical. De hecho, como la acción de $G$ es libre, la aplicación $\xi\mapsto \sigma_y(\xi)$ es un isomorfismo $\sigma_y: \gg \rightarrow V_y$ para cada $y$. Podemos ver entonces cuál es la relación entre la acción $R_{g,*}$ sobre $V_y$ inducida por la acción del grupo $G$ sobre $P$ y la representación adjunta $\ad_g$. Por definición,
	\begin{align*}
	  R_{g,*}\sigma_y(\xi)&=\left.\frac{d}{dt}\right|_{t=0}R_g(y\cdot \exp(t\xi))=\left.\frac{d}{dt}\right|_{t=0}(y\cdot \exp(t\xi)g)=\left.\frac{d}{dt}\right|_{t=0}(y\cdot gg^{-1}\exp(t\xi)g)\\ &=\left.\frac{d}{dt}\right|_{t=0}(y\cdot g\exp(t\ad_{g^{-1}}\xi))=\sigma_{y\cdot g}(\ad_{g^{-1}}\xi).
	\end{align*}
	Por tanto, $R_{g,*}\sigma(\xi)=\sigma(\ad_{g^{-1}}\xi)$.
	      
	El subespacio horizontal $H_y\subset T_yP$ es un subespacio vectorial que podemos ver como el conjunto de ceros de $k=\dim G$ ecuaciones lineales $T_yP \rightarrow \RR$ o, lo que es lo mismo, como el núcleo de $k$ $1$-formas en $y$. Podemos pensar en cada una de estas $1$-formas como las componentes de una $1$-forma $\omega$ con valores en un espacio vectorial $k$-dimensional, concretamente, como $\omega$ ha de anular vectores horizontales, podemos definirla por lo que sucede en el espacio $V_y$ que, como ya hemos visto, se identifica de forma natural con el álgebra de Lie $\gg$. Así, definimos:
	\begin{defn}
	  La \emph{$1$-forma de conexión} de una conexión $H\subset TP$ es la $1$-forma con valores en $\gg$, $\omega\in \Omega^1(P;\gg)$ definida por
	  \begin{equation*}
	    \omega(Y)=
	    \begin{cases}
	      \xi & \text{si } Y=\sigma(\xi), \\
	      0 & \text{si } Y \text{ es horizontal.}
	    \end{cases}
	  \end{equation*}
	  La relación entre la acción del grupo en $P$ y la representación adjunta también se manifiesta en la conexión:
	  \begin{prop}
	    La $1$-forma de conexión cumple
	    \begin{equation*}
	      R^*_g\omega=\ad_{g^{-1}}\circ \omega.
	    \end{equation*}
	  \end{prop}
	  \begin{proof}
	    Sea un vector $Y_y\in H_y$, de modo que $\omega(Y_y)=0$. Por la $G$-invariancia de $H$, $R_{g,*}Y_y\in H_{y\cdot g}$, luego $R_g^* \omega(Y_y)=0$ y se satisface trivialmente la igualdad. Por otro lado, si $Y_y=\sigma_y(\xi)$ para cierto $\xi \in \gg$, entonces
	    \begin{equation*}
	      R_g^*\omega(\sigma(\xi))=\omega(R_{g,*}\sigma(\xi))=\omega(\sigma(\ad_{g^-1}\xi))=\ad_{g^{-1}}\xi.
	    \end{equation*}
	  \end{proof}
	\end{defn}
	Recíprocamente, una $1$-forma de estas características induce automáticamente la conexión $H=\ker \omega$.

	Hay otra forma de pensar, más «física», que nos permite ver las conexiones como formas sobre el espacio base $B$ de un fibrado principal $p:P\rightarrow B$, en vez de sobre el espacio total $P$. Supongamos que $\mathcal{U}$ es un recubrimiento de $B$ por abiertos trivializantes y consideremos una familia de secciones locales $\left\{ s_U:U\rightarrow p^{-1}(U):U\in \mathcal{U} \right\}$. Si tenemos una $1$-forma de conexión $\omega \in \Omega(P;\gg)$ podemos considerar las pullback de $\omega$ por las secciones locales, lo que da una familia de $1$-formas en $B$ con valores en $\gg$: 
	\begin{equation*}
	  \left\{ A_U=s_U^*\omega \in \Omega^1(U;\gg): U\in \mathcal{U} \right\}.
	\end{equation*}
	Estas $1$-formas se suelen denominar un \emph{campo gauge} en $B$.

	\begin{prop}
	  La restricción de la $1$-forma de conexión $\omega$ a $p^{-1}(U)$, para un abierto trivializante $U$, es igual a
	  \begin{equation*}
	    \omega_U=\ad_{g_U^{-1}}\circ p^*A_U+ g_U^*\theta,
	  \end{equation*}
	  con $\theta$ la $1$-forma de Maurer-Cartan.
	\end{prop}
	\begin{proof}
	  En primer lugar vamos a ver que la igualdad se cumple en la imagen de la sección $s_U$. En efecto, si $x\in U$ e $y=s_U(x)$, podemos descomponer
	  \begin{equation*}
	    T_yP=\im(s_U\circ p)_* \oplus V_y,
	  \end{equation*}
	  de modo que cada $Y_y\in T_yP$ puede escribirse de forma única como $Y_y=(s_U)_*p_*(Y_y)+Y_y^v$, para cierto vector vertical $Y_y^v\in V_y$. Ahora, como $g_U(s_U(x))=1_G$, tenemos
	  \begin{align*}
	    \omega_U(Y)&=(p^*s_U^*\omega)(Y)+(g_U^*\theta_{1_G})(Y)=\omega(s_{U,*}p_* Y) + \theta_{1_G}(g_{U,*}Y)\\ &=\omega(s_{U,*}p_* Y) + \theta_{1_G}(g_{U,*}Y^v)=\omega(s_{U,*}p_* Y) + \omega(Y^v)=\omega(Y).
	  \end{align*}
	  Veamos ahora que ambos términos se transforman igual bajo la acción por la derecha de $G$:
	  \begin{align*}
	    R_g^*(\omega_U)_{y\cdot g}&=\ad_{g_U (y\cdot g)^{-1}}\circ R_g^*p^*s_U^*\omega+R_g^*g_U^*\theta=\ad_{(g_U (y)g)^{-1}}\circ R_g^*p^*s_U^*\omega+g_U^*R_g^*\theta\\&=\ad_{g^{-1}g_U (y)^{-1}}\circ p^*s_U^*\omega+g_U^*(\ad_{g^{-1}}\circ\theta)=\ad_{g^{-1}}\circ(ad_{g_U(y)^{-1}}\circ p^*s_U^*\omega+g_U^*\theta)\\&=\ad_{g^{-1}}\circ (\omega_U)_y.
	  \end{align*}
	  Por tanto, la igualdad se cumple en todo $p^{-1}(U)$.
	\end{proof}
	Ahora, como $\omega$ está definida globalmente, tenemos que $\omega_U=\omega_V$ en $p^{-1}(U\cap V)$. Esto nos permite obtener una relación entre $A_U$ y $A_V$ en $U\cap V$:
	\begin{equation*}
	  A_U=s_U^*\omega_U=s_U^*\omega_V=s_U^*(\ad_{g_V(s_U)^{-1}}\circ p^* A_V + g_V^*\theta)= \ad_{g_{UV}}\circ A_V + g_{VU}^*\theta.
	\end{equation*}

	Para cerrar la sección, vamos a ver cómo es posible obtener una noción de transporte paralelo a partir de una conexión. En primer lugar, consideremos $p:P\rightarrow B$ un fibrado principal con una conexión $H\subset TP$ y tomemos $x\in B$ e $y\in F_x$. Como $p_*:T_yP \rightarrow T_xB$ es sobreyectiva y su núcleo es precisamente $V_y$, tenemos que a cada vector $X_x\in T_xB$ le podemos asignar un único vector horizontal $X_y^h\in H_y$ tal que $p^*(X_y^h) =X_x$. Este nuevo vector se denomina el \emph{levantamiento horizontal} de $X_x$. 
	\begin{prop}
	  Sea $\omega$ la $1$-forma de conexión. Si tenemos un camino diferenciable $\gamma:[0,1]\rightarrow B$ y fijamos un punto $y\in F_{\gamma(0)}$, entonces existe un camino diferenciable $\gamma^h:[0,1]\rightarrow P$ tal que $\dot{\gamma}^h(t)\in H_{\gamma^h(t)}$, $\gamma^h(0)=y$ y $p\circ \gamma^h=\gamma$.
	\end{prop}

	\begin{proof}
	  Dado $\gamma:[0,1]\rightarrow B$, podemos considerar el \emph{fibrado pull-back}, con espacio total
	  \begin{equation*}
	    \gamma^*P=\left\{ (t,y)\in [0,1]\times P | \gamma(t)=p(y) \right\}
	  \end{equation*}
	  con $p':\gamma^*P \rightarrow [0,1]$ tal que $p(t,y)=t$. Consideramos la aplicación $\Phi$ dada por el diagrama
	  \begin{center}
	    \begin{tikzcd}
\gamma^*P	      \arrow{r}{\Phi}\arrow{d}[anchor=east]{p'} & P\arrow{d}[anchor=west]{p} \\ 
\left[0,1\right] \arrow{r}[anchor=south]{\gamma} & B
	     \end{tikzcd}
	   \end{center}
	   y el pullback de la $1$-forma de conexión, $\Phi^*\omega$, que a su vez nos da una conexión en $\gamma^*P$. Consideramos el levantamiento horizontal $\frac{\partial}{\partial t}^h$ con respecto a $\Phi^*\omega$ y lo integramos, de modo que existe una única curva horizontal $\eta:[0,1]\rightarrow \gamma^*P$ con $\eta(0)=\Phi^{-1}(y)$. Como esta curva es horizontal, $\gamma^h=\Phi \circ \eta$ es también horizontal y $\gamma^h(0)=y$ por construcción.
	\end{proof}
	Vemos entonces como este levantamiento nos da una manera de «conectar las fibras», explícitamente:
	\begin{align*}
	  F_{\gamma(0)}&\longrightarrow F_{\gamma(1)}\\ 
	    y &\longmapsto \gamma^h(1). 
	  \end{align*}
	  A partir de esta definición es posible entonces construir un transporte paralelo y una derivada de secciones, como planteábamos en la motivación al inicio de esta sección.
	  \begin{defn}
	    Sea $P\rightarrow B$ un fibrado principal con una $1$-forma de conexión $\omega$. Se define la \emph{derivada covariante} de una forma $\alpha\in \Gamma^k(P,\gg)$ como la $(k+1)$-forma con valores en $\gg$
	    \begin{equation*}
	      D_\omega \alpha(Y_1,\dots,Y_k)=d\alpha(Y_1^h,\dots,Y_k^h).
	    \end{equation*}
	  \end{defn}

	\section{La curvatura de una conexión}
	Sea $p:P\rightarrow B$ un fibrado principal y $H\subset TP$ una conexión en $P$ con $1$-forma de conexión $\omega$. Por definición, dado $y\in P$, $T_yP=V_y\oplus H_y$, de modo que cualquier vector $Y_y \in T_yP$ se descompone de forma única como $Y_y=Y_y^v+Y_y^h$, con $Y_y^v\in V_y$ e $Y_y^h\in H_y$.  
	\begin{defn}
	  Se define la \emph{curvatura} $\Omega \in \Omega^2(P;\gg)$ de la conexión $H$ como $\Omega=D_\omega \omega$, es decir
	  \begin{equation*}
	    \Omega(Y_y,Z_y)=d\omega(Y_y^h,Z_y^h),
	  \end{equation*}
	  con $Y_y, Z_y \in T_y P$ e $y\in P$.
	\end{defn}

	La interpretación geométrica de la curvatura es clara si consideramos
	\begin{equation*}
	  \Omega(Y,Z)=d\omega(Y^h,Z^h)=Y^h\omega(Z^h)-Z^h\omega(Y^h)-\omega([Y^h,Z^h])=-\omega([Y^h,Z^h]),
	\end{equation*}
	ya que $\omega$ se anula en los vectores horizontales. De aquí deducimos que $\Omega$ se anula si y sólo si $[Y^h,Z^h]$ es horizontal, es decir, si la distribución $H$ es integrable. Tenemos entonces que la curvatura en cierto modo «registra el fallo en la integrabilidad de $H$».
	      
	Vamos a hallar ahora dos fórmulas con la curvatura.
	\begin{prop}[Ecuación de estructura]
	  \begin{equation*}
	    \Omega=d\omega + [\omega,\omega]
	  \end{equation*}
	  (actuando sobre vectores $Y,Z$, tenemos $\Omega(Y,Z)=d\omega(Y,Z)+[\omega(Y),\omega(Z)]$).
	\end{prop}
	\begin{proof}
	  Dividimos la demostración en tres casos:
	  \begin{itemize}
	    \item Si $Y,Z$ son horizontales, $\omega(Y)=\omega(Z)=0$ y, como $Y=Y^h$ y $Z=Z^h$, la igualdad se cumple trivialmente.
	    \item Si $Y,Z$ son verticales, podemos suponer $Y=\sigma(\xi)$ y $Z=\sigma(\eta)$, con $\xi, \eta \in \gg$. Como $Y^h=Z^h=0$, basta ver que $d\omega(Y,Z)=-[\omega(Y),\omega(Z)]$. Pero
	      \begin{align*}
		d\omega(Y,Z)& =Y\omega(Z)-Z \omega(Y)-\omega([Y,Z])=-\omega([Y,Z])=-[\omega(Y),\omega(Z)],
	      \end{align*}
	      ya que $[X,Y]=[\sigma(\xi),\sigma(\eta)]=\sigma[\xi,\eta]$, $Y \omega(Z)=Y(\eta)=0$ y $Z \omega(Y)=Z(\xi)=0$.
	    \item Si $Y$ es horizontal y $Z=\sigma(\xi)$ es vertical, $\omega(Y)=0$ e $Y\omega(Z)=Y(\xi)=0$. Además, como $Z^h=0$, $\omega([Y^h,Z^h])=0$. Por tanto, la igualdad se reduce a ver que $-\omega([Y,Z])=0$, pero esto es cierto porque 
	      \begin{equation*}
		[Z,Y]=L_Z Y=\left.\frac{d}{dt}\right|_{t=0} R_{\exp(-t\xi),*}W=\left.\frac{d}{dt}\right|_{t=0} W_t=W,
	      \end{equation*}
	      con $W_t$ y $W$ vectores horizontales, por la $G$-equivariancia de la conexión. Por tanto, $-\omega([Y,Z])=\omega(W)=0$ por ser $W$ horizontal.
	  \end{itemize}

	\end{proof}

	\begin{prop}[Identidad de Bianchi]
	  Se tiene que $D_\omega \Omega=0$, es decir
	  \begin{equation*}
	    d\Omega(Y^h,Z^h,W^h)=0
	  \end{equation*}
	  para cualesquiera vectores $Y,Z,W$.
	\end{prop}
	\begin{proof}
	  Por la ecuación de estructura $\Omega=d\omega + [\omega,\omega]$, luego
	  \begin{equation*}
	    d\Omega=d[\omega,\omega]=[d\omega,\omega]-[\omega,d\omega].
	  \end{equation*}
	  Ahora, recordemos que la notación del corchete en las formas lo que representaba es
	  \begin{equation*}
	    [d\omega,\omega](Y,Z,W)=[d\omega(Y,Z),\omega(W)]+[d\omega(W,Y), \omega(Z)]+[d\omega(Z,W),\omega(Y)]
	  \end{equation*}
	  de modo que $d\omega(Y^h,Z^h,W^h)=0$ porque $\omega$ se anula en los vectores horizontales.
	\end{proof}

	Igual que hacíamos con las conexiones, podemos ver la curvatura de una conexión como una colección de $2$-formas sobre el espacio base $B$. Así, si $\mathcal{U}$ es un recubrimiento de $B$ por abiertos trivializantes y consideramos una familia de secciones locales $\left\{ s_U:U\rightarrow p^{-1}(U):U\in \mathcal{U} \right\}$, definimos la \emph{fuerza de campo gauge} como la familia
	\begin{equation*}
	  \left\{ F_U=s_U^* \Omega \in \Omega^2(U; \gg):U\in \mathcal{U} \right\}.
	\end{equation*}
	De la ecuación de estructura tenemos
	\begin{equation*}
	  F_U=dA_U+ [A_U,A_U].
	\end{equation*}
	Además, en las intersecciones $U\cap V$ la fuerza de campo gauge cambia como 
	\begin{equation*}
	  F_U=\ad_{g_{UV}}\circ F_V.
	\end{equation*}
	    
		  \section{Fibrados asociados}
Supongamos un grupo de Lie $G$ que actúa diferenciablemente en una variedad diferenciable $F$ por la izquierda y consideremos un fibrado principal $p:P\rightarrow B$ con fibra $G$.
\begin{defn}
  Se define el \emph{fibrado asociado a $p:P\rightarrow B$ via la acción de $G$ en $F$}, y se denota por $P\times_G F$, como el cociente $(P\times F)/G$ por la acción
  \begin{align*}
     (P\times F) \times G&\longrightarrow P\times F\\ 
     \left( (y,f),g \right) &\longmapsto (y\cdot g,g^{-1}\cdot f), 
    \end{align*}
    con la aplicación 
    \begin{align*}
      p_F :P\times_G F&\longrightarrow B\\ 
      [(y,f)] &\longmapsto p(y). 
      \end{align*}
\end{defn}

Veamos que esta definición da un fibrado con fibra $F$. Las trivializaciones de $P$, $\varphi_U:p^{-1}(U) \rightarrow U\times G$ inducen trivializaciones del fibrado asociado
\begin{align*}
  \varphi_U^F :p_F^{-1}(U)\cong U\times G \times F&\longrightarrow U\times F\\ 
  (x,g,f) &\longmapsto (x,gf) .
  \end{align*}
  Ademaś, las funciones de transición de $P\rightarrow B$,
  \begin{align*}
    \psi_{UV} :(U\cap V) \times G&\longrightarrow (U\cap V) \times G\\ 
    (x,g) &\longmapsto (x,h_{UV}g), 
    \end{align*}
    con $h_{UV}\in C^{\infty}(U\cap V,G)$ inducen funciones de transición en el fibrado asociado
    \begin{align*}
      \psi_{UV}^F :(U\cap V)\times F&\longrightarrow (U\cap V)\times F\\ 
      (x,f) &\longmapsto (x,h_{UV}(x)\cdot f). 
      \end{align*}

      \begin{ejemplo}
	Consideremos el fibrado principal $\SF^1 \rightarrow \SF^1$ del Ejemplo \ref{bormob}. Sea la fibra $[-1,1]$ y $\ZZ_2=\left\{ -1,1 \right\}$ actuando sobre $[-1,1]$ como $(f,\pm 1) \rightarrow \pm f$. Entonces su fibrado asociado $\SF^1\times_{\ZZ_2} [-1,1]\rightarrow \SF^1$ es la cinta de Moebius, vista como en el Ejemplo \ref{moebius}. Sin embargo, si consideramos la acción trivial $f\rightarrow f$, entonces el fibrado es el trivial, en ese caso $\SF^1\times_{\ZZ^2}[-1,1]$ es simplemente un cilindro.
	\qed
      \end{ejemplo}

      \begin{ejemplo}
	Sean $M$ una variedad diferenciable de dimensión $n$ y $L(M)\rightarrow M$ el fibrado de referencias. Consideramos la fibra $\RR^n$ con la acción canónica de $\mathrm{GL}(n,\RR)$ sobre $\RR^n$. Tenemos entonces un isomorfismo
	\begin{align*}
	  L(M)\times_{\mathrm{GL}(n,\RR)} \RR^n&\longrightarrow TM\\ 
	   [(\psi, \ve{v})] &\longmapsto \psi(\ve{v}), 
	  \end{align*}
	  donde vemos $\psi$ como un isomorfismo $\psi:\RR^n \rightarrow T_{p(\psi)}M$. Esta aplicación está bien definida ya que si tomo otro representante de la misma clase, $(\psi \cdot A, A^{-1} \cdot \ve{v})$, con $A\in \mathrm{GL}(n,\RR)$, entonces $\psi\cdot A (A^{-1} \cdot \ve{v})= \psi(A A^{-1} \ve{v})=\psi(\ve{v})$. 

	  Más generalmente, si $E\rightarrow B$ es un fibrado vectorial cualquiera con fibra un espacio vectorial $V$, entonces podemos considerar el fibrado $L(E)\rightarrow B$ de las bases de las fibras de $E$, que es un fibrado principal con fibra $\mathrm{Aut}(V)$ y tal que $E\cong L(E)\times_{\mathrm{Aut}(V)} V$.
	  \qed
      \end{ejemplo}

      Sea ahora $P\rightarrow B$ un fibrado principal con fibra $G$. De especial interés para nosotros es el fibrado asociado a la representación adjunta $\ad:G\rightarrow \mathrm{Aut}(\gg)$, que naturalmente proviene de una acción de $G$ en $\gg$. Este fibrado $\ad\ P:=P\times_G \gg$ se denomina el \emph{fibrado adjunto} de $P\rightarrow B$. El fibrado adjunto nos permite ver la curvatura de una conexión como una $2$-forma definida, no sobre el espacio total, sino sobre la base. Así, si tenemos una conexión en $P$ y $\Omega$ es su curvatura, podemos definir la $2$-forma con valores en $\ad\ P$, $\tilde{\Omega}\in \Omega(B,\ad\ P)$ tal que
      \begin{equation*}
	\tilde{\Omega}_x(X_x,Y_x)=[(y,\Omega_y(X_y^h,Y_y^h))],
      \end{equation*}
      donde $y\in F_x$, $X_x,Y_x \in T_xB$ y $X_y^h,Y_y^h$ son sus levantamientos horizontales a $T_y P$. Esta $2$-forma está bien definida, ya que si tomamos otro punto $y\cdot g \in F_x$ entonces
      \begin{align*}
	[(y\cdot g,\Omega_{y\cdot g}(X^h_{y\cdot g},Y^h_{y\cdot g})]&=[(y\cdot g, (R^*_g\Omega)_y(X^h_y,Y^h_y))]\\ &=[(y\cdot g,\ad_{g^{-1}}(\Omega_y(X_y^h,Y_y^h)))]\\&=[(y,\Omega_y(X^h_y,Y^h_y))].
      \end{align*}

      La otra gran ventaja de pensar en fibrados asociados es que éstos nos permiten vislumbrar la generalidad de los fibrados principales (de ahí su nombre). Sea $E\rightarrow B$ un fibrado con fibra $F$ y cuyas funciones de transición forman un grupo de Lie $G<\mathrm{Homeo}(F)$. Podemos definir un fibrado principal con fibra $G$ como sigue. Consideramos un recubrimiento $\mathcal{U}$ de $B$ por entornos trivializantes y consideramos el fibrado principal con fibra $G$ generado por este recubrimiento y las funciones de transición $g_{UV}\in G$ del fibrado original. Denotamos este fibrado como $P(E)$ y se puede comprobar que $E$ es isomorfo precisamente a $P(E)\times_G F$.

	\section{Clases características}
	La idea fundamental que motiva la introducción de las clases características surge del teorema de Gauss-Bonnet. Recordemos que este teorema nos dice que si $(M,g)$ es una superficie riemanniana conexa, orientada y compacta, entonces
	\begin{equation*}
	  \int_M K v_g = 2\pi \chi(M),
	\end{equation*}
	donde $K$ es la curvatura de Gauss de la superficie, $v_g$ la forma de área y $\chi(M)$ la característica de Euler de la superficie. De lo que es importante darse cuenta cuando se observa esta igualdad es de que se está identificando una expresión \textit{geométrica}, la del lado izquierdo, con otra \textit{topológica}, el lado derecho de la ecuación. Las clases características aparecen en el contexto de buscar identidades de este tipo en casos más generales. 

	Supongamos que $M$ es una variedad diferenciable de dimensión $n>2$ y que tenemos un fibrado principal $P\rightarrow M$. Si fijamos una conexión en $P$, la curvatura de esta conexión puede verse como una $2$-forma en $M$ con valores en $\ad\ P$, de modo que no es algo que en principio podamos integrar a lo largo de toda la variedad. Para solventar esto, si $n=2k$ es par podemos considerar la $n$ forma $\Omega\wedge \overset{(k)}{\cdots} \wedge \Omega$. Además, podemos tratar de encontrar funciones $f: \ad\ P \rightarrow \RR$, de modo que $f(\Omega\wedge \overset{(k)}{\cdots} \wedge \Omega)$ es una función en $M$ con valores reales.

	Para esta construcción, consideremos el conjunto $S^k(\gg)$ de todas las aplicaciones multilineales simétricas de la forma
	\begin{align*}
	  f :\gg \times \overset{(k)}{\cdots} \times \gg &\longrightarrow \RR.
	  \end{align*}
	  Por otra parte, decimos que una función $F:\gg \rightarrow \RR$ es un \emph{polinomio homogéneo de grado $k$ en $\gg$} si, fijado un isomorfismo $\gg \rightarrow \RR^m$, con $m=\dim \gg$, entonces el polinomio $F(\xx^1,\dots,\xx^m)$ que hace el siguiente diagrama conmutativo
	  \begin{center}
	    \begin{tikzcd}
\gg	      \arrow{r}{F}\arrow{rd} & \RR \\ 
&\RR^m \arrow{u}[anchor=west]{F(\xx^1,\dots,\xx^m)}
	     \end{tikzcd}
	   \end{center}
	   es un polinomio homogéneo de grado $k$, $F(\xx^1,\dots,\xx^m)\in \RR[\xx^1,\dots,\xx^n]_k$. Denotamos por $\RR[\gg]_k$ al conjunto de los polinomios homogéneos de grado $k$ en $\gg$. Hay una biyección entre $S^k(\gg)$ y $\RR[\gg]_k$ que a cada $f\in S^k(\gg)$ le asocia el polinomio homogéneo $F_f(\xi)=f(\xi,\dots,\xi)$ y a cada $P\in \RR[\gg]_k$ una función multilineal simétrica $f_F$, su \emph{polarización}:
	   \begin{equation*}
	     f_F(\xi_1,\dots,\xi_k)=\frac{1}{k!}\sum_{i=1}^k (-1)^{k-i}\sum_{\substack{j_1+\cdots+j_k=i \\ j_\alpha\in \left\{ 0,1 \right\}}} F(j_i\xi_1+\cdots j_k \xi_k).
	   \end{equation*}
	   En conjunto, hemos obtenido una biyección entre $S(\gg)=\bigoplus_kS^k(\gg)$ y $\RR[\gg]=\bigoplus_k \RR[\gg]_k$.

	   Como la curvatura toma valores en el fibrado adjunto $\ad\ P$, nos restringimos a aquellas funciones simétricas que sean invariantes respecto de la acción adjunta. Denotamos $I^k(\gg)\subset S^k(\gg)$ al conjunto de las aplicaciones multilineales simétricas tales que
	   \begin{equation*}
	     f(\ad_g \xi_1,\dots,\ad_g \xi_k)=f(\xi_1,\dots,\xi_k)
	   \end{equation*}
	   para cualquier $g\in G$. Una aplicación $f$ de esta forma (o su correspondiente en $\RR[\gg]$) se llama un \emph{polinomio invariante}.

	   \begin{thm}[Construcción de Weil de las clases características]
	     Sea $p:P\rightarrow B$ un fibrado principal con fibra $G$, $\omega$ una $1$-forma de conexión en $P$ con curvatura $\Omega$ y $f\in I^k(\gg)$. La $2k$-forma $f( \Omega\wedge \overset{(k)}{\cdots} \wedge \Omega)$ en $P$ definida por
	     \begin{equation*}
	       f(\Omega\wedge \overset{(k)}{\cdots} \wedge \Omega)(Y_1,\dots,Y_{2k})=\frac{1}{2k!}\sum_{\sigma \mathfrak{S}_{2k}}(-1)^{\sigma}f(\Omega(Y_{\sigma(1)},Y_{\sigma(2)}),\dots,\Omega(Y_{\sigma(2k-1)},Y_{\sigma(2k)}))
	     \end{equation*}
	     tiene las siguientes propiedades:
	     \begin{enumerate}
	       \item[$(1)$] Se puede proyectar (es la pullback por $p$ de una $2k$-forma en $B$).
	       \item[$(2)$]Es cerrada.
	       \item[$(3)$] La clase de cohomología de su proyección en $B$ no depende de la elección de la conexión $\omega$. Esta clase se llama la \emph{clase característica del fibrado $P\rightarrow B$ asociada a $f$}.
	     \end{enumerate}
	   \end{thm}
	   \begin{proof}
	Para probar $(1)$, vamos a dar explícitamente una forma en $B$ cuya pullback por $p$ es $f( \Omega\wedge \overset{(k)}{\cdots} \wedge \Omega)$. Dada $f\in I^k(\gg)$, definimos la aplicación
	\begin{align*}
	  \tilde{f} :\ad\ P \times \cdots \times \ad\ P&\longrightarrow \RR\\ 
	  ([(y,\xi_1)],\dots,[(y,\xi_k)]) &\longmapsto f(\xi_1,\dots,\xi_k), 
	  \end{align*}
	  que está bien definida ya que 
	  \begin{equation*}
	    \tilde{f}([(yg,\ad_{g^{-1}}\xi_1)],\dots,[(yg,\ad_{g^{-1}}\xi_k)])=f(\ad_{g^{-1}}\xi_1,\dots,\ad_{g^{-1}}\xi_k)=f(\xi_1,\dots,\xi_k).
	  \end{equation*}
	  Esto permite definir una $2$-forma $\tilde{f}(\tilde{\Omega},\dots,\tilde{\Omega})$ en $B$ como
	  \begin{equation*}
	    \tilde{f}(\tilde{\Omega},\dots,\tilde{\Omega})(X_1,\dots,X_{2k})=\frac{1}{2k!}\sum_{\sigma \in \mathfrak{S}_{2k}}(-1)^{\sigma} \tilde{f}(\tilde{\Omega}(X_{\sigma(1)},X_{\sigma(2)}),\dots,\tilde{\Omega}(X_{\sigma(2k-1)},X_{\sigma(2k)})).
	  \end{equation*}
	  Ahora, si consideramos los campos $Y_i \in \mathfrak{X}(P)$ y denotamos $X_i=p_*X_i$, entonces la pullback de la forma construida es
	  \begin{align*}
	    p^*(\tilde{f}(\tilde{\Omega},\dots,\tilde{\Omega}))(Y_1,\dots,Y_{2k})&=\frac{1}{2k!}\tilde{f}(\tilde{\Omega},\dots,\tilde{\Omega})(p_*Y_1,\dots,p_*Y_{2k})\\&=\frac{1}{2k!}\sum_{\sigma \in \mathfrak{S}_{2k}}(-1)^{\sigma} \tilde{f}(\tilde{\Omega}(X_{\sigma(1)},X_{\sigma(2)}),\dots,\tilde{\Omega}(X_{\sigma(2k-1)},X_{\sigma(2k)}))\\
	    &=\frac{1}{2k!}\sum_{\sigma \in \mathfrak{S}_{2k}}(-1)^{\sigma} f(\Omega(Y_{\sigma(1)},Y_{\sigma(2)}),\dots,\Omega(Y_{\sigma(2k-1)},Y_{\sigma(2k)}))\\
	    &= f(\Omega \wedge \overset{(k)}{\cdots} \wedge \Omega)(Y_1,\dots,Y_{2k}).
	  \end{align*}
	  
	  Para probar $(2)$ basta notar que por definición $f(\Omega \wedge \overset{(k)}{\cdots} \wedge \Omega)$ se anula en los vectores verticales, luego $df(\Omega \wedge \overset{(k)}{\cdots} \wedge \Omega)=D_\omega f(\Omega \wedge \overset{(k)}{\cdots} \wedge \Omega)$, ahora, por la identidad de Bianchi:
	  \begin{align*}
	    D_\omega f(\Omega \wedge \overset{(k)}{\cdots} \wedge \Omega)
	    =f(D_\omega\Omega \wedge \Omega \wedge \overset{(k)}{\cdots} \wedge \Omega)+\cdots+f(\Omega \wedge \Omega \wedge \overset{(k)}{\cdots} \wedge D_\omega \Omega)=0.
	  \end{align*}

	  Finalmente, vamos a probar $(3)$. Sean $\omega_0$ y $\omega_1$ dos $1$-formas de conexión en $P$ y denotemos $\alpha=\omega_1-\omega_0 \in \Omega^1(P,\gg)$. Podemos considerar la familia uniparamétrica de conexiones $\omega_t=\omega_0+t\alpha$, con curvaturas
	  \begin{equation*}
	    \Omega_t=d\omega_t + [\omega_t,\omega_t]=d\omega_0 + td\alpha + [\omega_0+t\alpha,\omega_0+t\alpha].
	  \end{equation*}
	  En particular,
	  \begin{equation*}
	    \frac{d}{dt}\Omega_t=d\alpha + [\alpha,\omega_t]+[\omega_t,\alpha]=D_{\omega_t}\alpha.
	  \end{equation*}

Queremos ver que $f(\Omega_1 \wedge \overset{(k)}{\cdots} \wedge \Omega_1)-f(\Omega_0 \wedge \overset{(k)}{\cdots} \wedge \Omega_0)=dh$ para cierta forma $h$, que queremos hallar. Vamos a construir esta $h$ como
\begin{equation*}
  h=k\int_0^1 f(\alpha \wedge \Omega_t \wedge \overset{(k-1)}{\cdots} \wedge \Omega_t) dt.
\end{equation*}
En efecto,
\begin{align*}
  dh =&k\int_{0}^1 df(\alpha\wedge \Omega_t \wedge \overset{(k-1)}{\cdots} \wedge \Omega_t)dt=k\int_{0}^1 D_{\omega_t}f(\alpha\wedge \Omega_t \wedge \overset{(k-1)}{\cdots} \wedge \Omega_t)dt \\
  =& k\int_{0}^1 f(D_{\omega_t}\alpha\wedge \Omega_t \wedge \overset{(k-1)}{\cdots} \wedge \Omega_t)dt+k\int_{0}^1 f(\alpha\wedge D_{\omega_t}\Omega_t \wedge \overset{(k-1)}{\cdots} \wedge \Omega_t)dt\\ &+k\int_{0}^1 f(\alpha\wedge \Omega_t \wedge \overset{(k-1)}{\cdots} \wedge D_{\omega_t}\Omega_t)dt
  = k\int_0^1 f\left( \frac{d\Omega_t}{dt} \wedge \Omega_t \wedge \overset{(k-1)}{\cdots} \wedge \Omega_t \right)dt\\
  =& \int_{0}^1 \frac{d}{dt} f(\Omega_t\wedge \overset{(k)}{\cdots} \wedge \Omega_t)dt = f(\Omega_1\wedge \overset{(k)}{\cdots} \wedge \Omega_1)-f(\Omega_0\wedge \overset{(k)}{\cdots} \wedge \Omega_0)
\end{align*}
  
De aquí concluimos que $[\tilde{f}(\tilde{\Omega},\dots,\tilde{\Omega})]\in H^{2k}(B,\RR)$ no depende de la elección de la conexión, como queríamos probar.
  \end{proof}
	   \begin{obs}
	     La construcción es exactamente igual si tomamos polinomios complejos, de modo que obtendríamos una clase característica en $H^{2k}(B,\CC)$.
	   \end{obs}

	   \begin{ejemplo}
	     Consideremos un fibrado principal $P\rightarrow B$ con fibra $\mathrm{GL}(n,\RR)$. El álgebra de Lie de la fibra es $\mathfrak{gl}(n,\RR)=\mathfrak{M}_n(\RR)$ el conjunto de las matrices $n\times n$ con coeficientes reales. Se puede probar que el conjunto de los polinomios invariantes está generado por los polinomios $\left\{ p_1,\dots,p_n \right\}$ definidos por
	     \begin{equation*}
	       \det(\lambda I-A)=\lambda^n+p_1(A)\lambda^{n-1}+\cdots+p_{n-1}(A)\lambda + p_n(A),
	     \end{equation*}
	     para $A\in \mathfrak{M}_n(\RR)$.

	     Análogamente, si consideramos el grupo $\mathrm{GL}(n,\RR)$, el conjunto de los polinomios invariantes está generado por los polinomios $\left\{ p_1,\dots,p_n \right\}$ definidos por
	     \begin{equation*}
	       \det(\lambda I+iA)=\lambda^n+p_1(A)\lambda^{n-1}+\cdots+p_{n-1}(A)\lambda + p_n(A).
	     \end{equation*}
	     Las clases características $c_k$ asociadas a cada uno de los polinomios $\frac{1}{(2\pi)^k}p_k$ se denominan las \emph{clases de Chern} del fibrado $P$.
	     \qed
	   \end{ejemplo}
	  
	   \begin{ejemplo}
	     Consideremos la esfera $\SF^2$, $U_N$ y $U_S$ los hemisferios norte y sur, respectivamente y el fibrado principal con fibra $U(1)$ definido por una función de transición
	     \begin{align*}
	       \psi :U_N\cap U_S&\longrightarrow U(1) .
	       \end{align*}
	       Nótese que $U_N\cap U_S$ es el ecuador de la esfera mientras que $U(1)\cong \SF^1$, de modo que podemos ver la función de transición como una aplicación $\psi:\SF^1 \rightarrow \SF^1$. Las aplicaciones $\SF^1\rightarrow \SF^1$ están determinadas salvo homotopía por su grado $k\in \ZZ$, y podemos tomar como representante de cada clase de homotopía la función $2\pi$-periódica $\psi(\phi)= k\phi$. Así, para cada grado $k$ tenemos un fibrado principal distinto $P_k\rightarrow \SF^2$. Como $U(1)$ es abeliano, la acción adjunta es trivial y todos los polinomios son invariantes, de modo que escogemos el polinomio $p(\xx)=\frac{1}{2\pi}\xx$. Consideramos la conexión definida por un campo gauge $A$ de la forma:
	       \begin{itemize}
		 \item en $U_N$, $A_N=0$,
		 \item en $U_N\cap U_S$,  $A_S=\psi A_N + \psi d\psi=\psi d\psi$ (que es la forma de Maurer-Cartan en la intersección) y lo extendemos de cualquier manera a todo $U_S$.
	       \end{itemize}
	       La curvatura vendrá dada por la fuerza de campo gauge $F=dA+[A,A]=dA$, donde hemos usado que $U(1)$ es abeliano. Por tanto, $p(F)=\frac{1}{2\pi}dA$ y
	       \begin{align*}
		 \int_{\SF^2} p(\Omega)&=\frac{1}{2\pi} \int_{U_N}dA_N+\frac{1}{2\pi}\int_{U_S}dA_S=\frac{1}{2\pi}\int_{U_S}dA_S\\ &=\frac{1}{2\pi}\int_{U_N\cap U_S}A_S|_{U_N\cap U_S}=\frac{1}{2\pi}\int_{U_N\cap U_S}\psi d\psi= \frac{1}{2\pi} \int_0^{2\pi} k\phi d\phi = k.
	       \end{align*}
	       Tenemos entonces que la primera clase de Chern $c_1$ clasifica los fibrados principales con fibra $U(1)$.
	   \end{ejemplo}

	   \appendix
	   \section{Una aplicación física: El campo electromagnético}
Para concluir el trabajo, veamos un caso donde los fibrados principales se pueden aplicar a la Física: el campo electromagnético. La teoría de los fibrados principales nos permite entender el campo electromagnético como un fibrado principal de fibra $U(1)$ (el grupo de transformaciones gauge del electromagnetismo) sobre el espacio-tiempo de Minkowski $\RR^4$. En este caso, el fibrado es siempre el trivial $P=\RR^4\times U(1)$. Podemos dar la conexión en el fibrado como un campo gauge $A\in \Omega(\RR^4,i\RR)$, donde estamos pensando en $i\RR$ como el álgebra de Lie de $U(1)$, en virtud de la asignación $i\phi \mapsto e^{i\phi}$. Es decir la conexión es de la forma $A=iA_{\mu}dx^{\mu}$ (a partir de aquí usamos el convenio de índices repetidos). La curvatura vendrá dada por la fuerza del campo $F=dA$ con componentes
	   \begin{equation*}
	     iF_{\mu\nu}=i\left( \frac{\partial A_\nu}{\partial x^\mu}-\frac{\partial A_\mu}{\partial x^\nu} \right).
	   \end{equation*}
	   La identidad de Bianchi da
	   \begin{equation*}
	     dF=0,
	   \end{equation*}
	   que en componentes es
	   \begin{equation*}
	     \partial_\lambda F_{\mu\nu}+\partial_\nu F_{\lambda\mu}+\partial_\mu F_{\nu\lambda}=0.
	   \end{equation*}
	   Identificando las componentes $F_{\mu\nu}$ con los campos eléctrico $\ve{E}$ y magnético $\ve{B}$ en la forma $E_i=F_{i0}$, $B_i=\frac{1}{2}\epsilon_{ijk}F_{jk}$, la igualdad anterior se reduce a las ecuaciones
	   \begin{equation*}
	     \nabla \times \ve{E}+ \frac{\partial \ve{B}}{\partial t}=0,\  \ \ \nabla \cdot \ve{B}=0.
	   \end{equation*}
	   Estas dos ecuaciones son el par «estático» de las ecuaciones de Maxwell, ya que, como hemos visto, vienen impuestas directamente por la estructura geométrica del espacio-tiempo, sin necesidad de tener en cuenta consideraciones dinámicas. El otro par de ecuaciones de Maxwell puede obtenerse también en el contexto de los fibrados principales e irán asociadas a una conexión específica, la que minimiza la acción de Yang-Mills.

	   De momento hemos considerado el campo electromagnético en el espacio-tiempo de Minkowski, que es topológicamente trivial, pero podemos considerar espacios base más exóticos. Es el caso, por ejemplo, del \emph{monopolo magnético} de carga magnética $g$ en el origen $\RR^3$, cuyo campo magnético es
	   \begin{equation*}
	     \ve{B}(\ve{x})=g\frac{\ve{x}}{\norm{\ve{x}}^3},
	   \end{equation*}
	   definido en $\RR^3-\left\{ 0 \right\}$. Es inmediato comprobar que, en efecto, $\nabla \cdot \ve{B} =0$, de modo que, en $\RR^3-\left\{ 0 \right\}$, $\ve{B}$ es una solución de las ecuaciones de Maxwell. De acuerdo con la correspondencia anterior, podemos ver el campo magnético como la forma 
	   \begin{equation*}
	     F=i\frac{g}{\norm{\ve{x}}^3}(x^1 dx^2\wedge dx^3-x^2 dx^1\wedge dx^3 + x^3 dx^1 \wedge dx^2).
	   \end{equation*}
	   De nuevo, volvemos a introducir la $i$ para ver $F$ como una forma con coeficientes en el álgebra de Lie de $U(1)$.
	   Más aún, como $\RR^3-\left\{ 0 \right\}$ es homotópicamente equivalente a la esfera $\SF^2$, podemos restringirnos simplemente a estudiar lo que pasa en la esfera unidad y estudiar ahí el campo $F$. Volvemos entonces al caso de estudiar fibrados principales con fibra $U(1)$ sobre $\SF^2$.

	   En coordenadas esféricas en la esfera, $F$ se expresa simplemente como
	   \begin{equation*}
	     F=ig\sin \theta d\theta \wedge d\phi.
	   \end{equation*}
	   Sabemos que el segundo grupo de cohomología de de Rham $H^2(\SF^2;\RR)$ es no nulo, así que en principio no cabe esperar una $1$-forma $A$ tal que $F=dA$. Sin embargo, sí que podemos encontrar estas formas localmente: en cada hemisferio $U_N$, $U_S$, definimos
	   \begin{equation*}
	     A_N=ig(1-\cos \theta)d\phi \ \ \ A_S=-ig(1+\cos \theta) d\phi.
	   \end{equation*}
	   Ahora, una función de transición $U_N\cap U_S \rightarrow U(1)$ puede verse simplemente como una función de la forma
	   \begin{align*}
	     \psi :\SF^1&\longrightarrow \SF^1\\ 
	     \phi &\longmapsto e^{i\varphi(\phi)},
	     \end{align*}
	     con $\varphi:\SF^1\rightarrow \RR$.
	     Los potenciales han de relacionarse ahora en la forma
	     \begin{equation*}
	       A_N=\psi^{-1}A_S\psi +\psi^{-1}d\psi=A_S+id\varphi.
	     \end{equation*}
	     Por tanto
	     \begin{equation*}
	       d\varphi = -i(A_N - A_S)=2g d\phi.
	     \end{equation*}
	     Ahora, podemos hallar el período de $\varphi$,
	     \begin{equation*}
	       \Delta \varphi = \int d\varphi = \int_{0}^{2\pi} 2g d\phi = 4\pi g.
	     \end{equation*}
	     Como $\varphi$ ha de ser $2\pi$-periódica, tenemos que 
	     \begin{equation*}
	       2g=\frac{\Delta \varphi}{2\pi}
	     \end{equation*}
	     ha de ser un número entero. Este resultado es lo que en Física se conoce como la \emph{condición de cuantización de Dirac}: las cargas magnéticas deben ser semienteras.

	     \nocite{*}
	     \bibliographystyle{plain}
	     \bibliography{biblio}

\end{document}


