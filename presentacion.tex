%\documentclass[mathserif]{beamer}
\documentclass{beamer}
%\documentclass[aspectratio=169]{beamer}
\beamertemplatenavigationsymbolsempty

\usepackage[utf8]{inputenc}
\usepackage[spanish]{babel}

\usepackage{amsmath}
\usepackage{amsfonts}
\usepackage{amssymb}
\usepackage{amsthm, mathtools}
%\usepackage{mathrsfs}
\usepackage{tikz-cd}
%\usepackage{eulervm}
\usepackage{libertine}
%\usepackage[scaled]{helvet}
\usepackage[libertine]{newtxmath}
%\usepackage{mathpazo}
%\usepackage{newtxmath}
\usepackage{anyfontsize}
%\usepackage{lmodern}
%\usepackage[mathscr]{eucal}

%\usepackage{fontspec}
%\setsansfont{Comic Sans MS}
%\setsansfont{Ubuntu}

\deftranslation[to=spanish]{Theorem}{Teorema}
\deftranslation[to=spanish]{Definition}{Definición}

\newtheorem{prop}{Proposición}
\newtheorem{lema}{Lema}

\newcommand{\pr}{\text{pr}}
\newcommand{\id}{\text{id}}
\newcommand{\ad}{\text{ad}}
\newcommand{\TT}{\mathbb{T}}
\newcommand{\SF}{\mathbb{S}}
\newcommand{\RR}{\mathbb{R}}
\newcommand{\CC}{\mathbb{C}}
\newcommand{\ZZ}{\mathbb{Z}}
\newcommand{\GG}{\mathfrak{g}}

\title{Introducción a los fibrados principales}
\author{Guillermo Gallego Sánchez}
\institute{Geometría de superficies topológicas}
\date{17 de diciembre de 2018}

\logo{\includegraphics[height=.7cm]{logogris}}

		      \let\emph\relax
		      \DeclareTextFontCommand{\emph}{\it\bfseries}

\begin{document}
\begin{frame}
  \maketitle
\end{frame}
\begin{frame}{Fibrados}
  \begin{columns}[T]
    \begin{column}{.25\textwidth}
	  \begin{center}
	\begin{tikzcd}[ampersand replacement=\&]
	    F \arrow{r} \& E\arrow{ddd}{p}    \\ 
	    \& \\
	    \& \\
	    \& B
	  \end{tikzcd}
	\end{center}
    \end{column}
    \begin{column}{.5\textwidth}
      Un \emph{fibrado} $(E,B,p,F)$ \pause consta de:
      \begin{itemize}
	\item \emph{base}: $B$ \pause
	\item \emph{espacio total}: $E$ \pause
	\item \emph{fibra}: $F$ \pause
	\item $p:E\rightarrow B$ aplicación continua \pause tal que $\forall x\in B$ $\exists U^x$ y una \emph{trivialización} \pause
	\begin{tikzcd}[ampersand replacement=\&]
	  p^{-1}(U) \arrow{rr}{\varphi_U} \arrow{rd}{p} \& \& U\times F\arrow{ld}{\pr_1}    \\ 
	    \& U \&
	  \end{tikzcd}
      \end{itemize}
    \end{column}
  \end{columns}
\end{frame}
\begin{frame}{Ejemplos}
  \begin{itemize}
    \item El \emph{fibrado trivial}:
	  \begin{center}
	\begin{tikzcd}[ampersand replacement=\&]
	    F \arrow{r} \& B\times F\arrow{d}{\pr_1}    \\ 
	    \& B
	  \end{tikzcd}
	\end{center}\pause
      \item La \emph{cinta de Moebius}:
	  \begin{center}
	\begin{tikzcd}[ampersand replacement=\&]
	    (0,1) \arrow{r} \& E\arrow{d}{\pr_1}    \\ 
	    \& \SF^1,
	  \end{tikzcd}
	\end{center}
	con $E=\left\{ (x,y):x\in \RR, y \in (0,1) \right\}/(x,y)\sim (x+1,1-y)$ y $p:E\rightarrow \SF^1$, viendo la base como $\SF^1=\left\{ x\in \RR \right\}/x\sim x+1$.
  \end{itemize}
\end{frame}

\begin{frame}{Funciones de transición}
  $E\rightarrow B$ fibrado, $x\in B$, $U^x$, $V^x$. 
  \pause
    \begin{center}
      \begin{tikzcd}[ampersand replacement=\&]
	U\cap V \times F \arrow{r}{\varphi^{-1}_U} \arrow{rd}{\pr_1} \& p^{-1}(U\cap V) \arrow{d}{p} \arrow{r}{\varphi_V}\& U\cap V \times F \arrow{ld}{\pr_1} 	\\ 
\&	U\cap V\&
      \end{tikzcd}
    \end{center}
    \pause

      \begin{align*}
	\varphi_V \circ \varphi_U^{-1} :(U\cap V) \times F&\longrightarrow (U\cap V)\times F\\ 
	(x,y) &\longmapsto (x,\psi_{UV}(x,y))
	\end{align*}

	\pause
	\begin{align*}
	  \psi_{UV,x} :F&\longrightarrow F\\ 
	  y &\longmapsto \psi_{UV}(x,y) 
	  \end{align*}
	\end{frame}

\begin{frame}{Funciones de transición}
  \begin{itemize}
\item \emph{Función de transición} entre $U$ y $V$:
	  \begin{align*}
	    g_{UV} :U\cap V&\longrightarrow \text{Homeo}(F)\\ 
	    x &\longmapsto \psi_{UV,x}. 
	    \end{align*}
	  \item Las funciones de transición cumplen la \emph{condición de cociclo}
	    \begin{equation*}
	      g_{UW}=g_{VW}\circ g_{UV}.
	    \end{equation*}
	    En particular, $g_{UU}=\id$ y $g_{UV}=g_{VU}^{-1}$.
	\item Con estructura adicional en las funciones de transición obtenemos otro tipo de fibrados (por ejemplo, fibrados vectoriales $\rightsquigarrow$ matrices de transición).
  \end{itemize}
\end{frame}

\begin{frame}{Isomorfismo de fibrados}
  $p:E\rightarrow B$, $p':E\rightarrow B'$ fibrados con fibra $F$. \pause

  Un \emph{isomorfismo de fibrados} entre $p:E\rightarrow B$ y $p:E'\rightarrow B'$ es un par de homeomorfismos $(f,\tilde{f})$ tales que \pause el diagrama
  \begin{center}
    \begin{tikzcd}[ampersand replacement=\&]
      E'      \arrow{rr}{\tilde{f}}\arrow{dd}[anchor=east]{p'} \&\& E\arrow{dd}[anchor=west]{p} \\ 
       \&\& \\
       B'\arrow{rr}[anchor=south]{f} \&\& B
     \end{tikzcd}
   \end{center}
   conmuta.

   \pause
    Las funciones de transición se relacionan por 
     \begin{equation*}
       g'_{U'V'}=f^{-1}_{VV'}\circ g_{UV} \circ f_{UU'},
     \end{equation*}
     con $f_{UU'}:U'\rightarrow \text{Homeo}(F)$.
\end{frame}

\begin{frame}{Obtención del fibrado desde las funciones de transición}
  $\mathcal{U}$ recubrimiento abierto de $B$, $G<\text{Homeo}(F)$ y $\left\{ g_{UV}:U\cap V\rightarrow G:U,V\in \mathcal{U} \right\}$ conjunto de funciones de transición de un fibrado $E\rightarrow B$.\pause
  
  Entonces $E\rightarrow B$ es isomorfo al fibrado $E'\rightarrow B$ con \pause
  \begin{itemize}
    \item espacio total
      \begin{equation*}
	E'=\bigsqcup_{U\in \mathcal{U}}(U\times F)/(x,y)\sim(x,g_{UV}(x)(y))
      \end{equation*}\pause
    \item la proyección
      \begin{align*}
	p :E'&\longrightarrow B\\ 
	[(x,y)] &\longmapsto  x.
	\end{align*}
  \end{itemize}
\end{frame}

\begin{frame}{Fibrados y cohomología de \v{C}ech}
  $\mathcal{U}$ recubrimiento abierto de $B$. $G<\text{Homeo}(F)$.
  
  \pause
  \begin{itemize}
    \item
  Un conjunto de funciones $\left\{ g_{UV}:U\cap V\rightarrow G:U,V \in \mathcal{U} \right\}$ es:
      un \emph{$1$-cociclo de \v{C}ech subordinado a $\mathcal{U}$ con coeficientes en $G$} si 
      \begin{equation*}
	g_{UW}=g_{VW}\circ g_{UV},
      \end{equation*}
      \pause
    \item Se llama \emph{primer grupo de cohomología de \v{C}ech subordinado a $\mathcal{U}$ con coeficientes en $G$} al cociente
      \begin{equation*}
	\check{H}(\mathcal{U},G)=\left\{ 1\text{-cociclos} \right\}/g_{UV}\sim (f^{-1}_{VV'}\circ g_{UV} \circ f_{UU'}).
      \end{equation*}
      \pause
    \item Tomando el límite directo por refinamiento del recubrimiento, tenemos el \emph{primer grupo de cohomología de \v{C}ech con coeficientes en $G$} 
      \begin{equation*}
	\check{H}^1(B,G)=\varinjlim_{\mathcal{U}} \check{H}^1(\mathcal{U},G).
      \end{equation*}
  \end{itemize}
\end{frame}

\begin{frame}{Secciones}
  \begin{columns}[T]
    \begin{column}{.25\textwidth}
	  \begin{center}
	\begin{tikzcd}[ampersand replacement=\&]
	  F \arrow{r} \& E\arrow{ddd}[anchor=east]{p}    \\ 
	    \& \\
	    \& \\
	    \& B\arrow[bend right]{uuu}{s}
	  \end{tikzcd}
	\end{center}
    \end{column}
    \begin{column}{.5\textwidth}
      Una \emph{sección} de un fibrado $p:E\rightarrow B$ es una aplicación continua $s:B\rightarrow E$ tal que $p\circ s=\id_B$.  \pause

      Una sección local es una sección definida en un abierto $U\subset B$. \pause
     
      Denotamos $\Gamma(E)$ al conjunto de las secciones de $E\rightarrow B$ y $\Gamma(U,E)$ al conjunto de las secciones locales definidas en un abierto $U\subset B$.
    \end{column}
  \end{columns}
\end{frame}
\begin{frame}{Fibrados principales}
  Un \emph{fibrado principal} $(P,B,p,G)$ consta de:
  \begin{itemize}
    \item una variedad diferenciable $P$, \pause
    \item un grupo de Lie $G$ actuando libremente por la derecha sobre $P$:
      \begin{align*}
	 P\times G&\longrightarrow P\\ 
	  (p,g) &\longmapsto p\cdot g, 
	\end{align*} \pause
      \item $B=P/G$ con una sumersión $p:P\rightarrow P/G$, que es la proyección canónica al cociente \pause y tal que $\forall x\in B$ $\exists U^x$ y una trivialización 
	\begin{center}
	\begin{tikzcd}[ampersand replacement=\&]
	  p^{-1}(U) \arrow{rr}{\varphi_U} \arrow{rd}{p} \& \& U\times G\arrow{ld}{\pr_1}    \\ 
	    \& U \&.
	  \end{tikzcd} 
	\end{center}\pause

	  Además, $\varphi_U(y)=(p(y),g_U(y))$ para cierta $g_U:p^{-1}(U)\rightarrow G$ con $g_{U}(y\cdot g)=g_U(y)\cdot g$.
  \end{itemize}
\end{frame}

\begin{frame}{Observaciones}
  \begin{itemize}
    \item Podemos pensar en un fibrado principal como en un fibrado sobre una variedad diferenciable cuya fibra es un grupo de Lie. \pause
      En efecto, para cada sección local $s_U:U\rightarrow P$ y para cada $y\in p^{-1}(x)$ existe un único elemento $g_U(y)\in G$ con $y=s_U(x)g_U(y)$. \pause
    \item Las funciones de transición son de la forma
      \begin{align*}
	 U\cap V \times G&\longrightarrow U\cap V\times G\\ 
	 (x,h) &\longmapsto (x,g_{UV}(x)h). 
	\end{align*}
	\pause
      \item Un fibrado principal admite una sección global si y sólo si es trivial.
  \end{itemize}
\end{frame}

\begin{frame}{Ejemplos}
  \begin{itemize}
    \item $P=B=\SF^1\subset \CC$ y 
      \begin{align*}
	p :\SF^1&\longrightarrow \SF^1\\ 
	  z &\longmapsto z^2.
	\end{align*}
	\pause
	Fibra $\ZZ_2$ con la acción
	\begin{align*}
	   \SF^1\times \ZZ_2&\longrightarrow \SF^1\\ 
	    (z,\pm 1) &\longmapsto \pm z. 
	  \end{align*}
	  \pause
	\item $p:\tilde{M}\rightarrow M$ recubridor universal. \pause
	  
	 Fibra $\pi_1(M)$ con la acción de monodromía: 
	\begin{align*}
	  \tilde{M}\times \pi_1(M)&\longrightarrow \tilde{M}\\ 
	  (y,g) &\longmapsto \tilde{\gamma}_g^y(1).
	  \end{align*}
  \end{itemize}
\end{frame}

\begin{frame}{Fibrado de referencias}
  El \emph{fibrado de referencias} sobre una variedad diferenciable $M$ tiene por espacio total
  \begin{equation*}
    L(M)=\left\{ \psi_x:\RR^n \rightarrow T_x M: x\in M,\ \psi_x \text{ es un isomorfismo lineal}  \right\}
  \end{equation*}
  \pause
  con la aplicación
  \begin{align*}
    p :L(M)&\longrightarrow M\\ 
      \psi_x &\longmapsto x. 
    \end{align*}
    \pause
    Su fibra es $\text{GL}(n,\RR)$\pause, con la acción
	\begin{center}
	\begin{tikzcd}[ampersand replacement=\&]
	  \RR^n \arrow{r}{A} \& \RR^n \arrow{r}{\psi_x} \& T_x M.
	  \end{tikzcd} 
	\end{center}
\end{frame}

\begin{frame}{Conexiones en fibrados principales}
  $p:P\rightarrow B$ con fibra $G$. $x\in B$ $y\in p^{-1}(x)$.
  \pause
  \begin{itemize}
    \item \emph{Subespacio vertical}: $V_y=\ker p_* \subset T_y Y$.
  \pause
    \item \emph{Campo vertical}: $X_y \in V_y$ $\forall y \in P$. El corchete de Lie de campos verticales es vertical.
  \pause
    \item $V\subset TP$ es \emph{$G$-invariante}: $\forall g\in G$, $R_{g,*}V_y=V_{y\cdot g}$.
  \pause
    \item Una \emph{conexión} en $P$ es $H\subset V$, $G$-invariante y tal que $TP=V\oplus H$.
  \end{itemize}
\end{frame}

\begin{frame}{El campo fundamental}
  \begin{align*}
    \sigma :\GG&\longrightarrow \mathfrak{X}(P)\\ 
      \xi &\longmapsto \sigma(\xi). 
    \end{align*}
    Este $\sigma(\xi)$ se llama \emph{campo fundamental} \pause y su valor es
    \begin{equation*}
      \sigma_y(\xi)=\left.\frac{d}{dt} \right|_{t=0}(y\cdot \exp(t\xi)).
    \end{equation*}
    \pause
    \begin{itemize}
      \item $p_*\sigma_y(\xi)=0$, luego $\sigma(\xi)$ es un campo vertical.
	\pause
      \item $\xi \mapsto \sigma_y(\xi)$ es un isomorfismo.
    \end{itemize}
\end{frame}

\begin{frame}{La $1$-forma de conexión}
  La \emph{$1$-forma de conexión} de una conexión $H\subset TP$ es $\omega\in \Omega^1(P;\GG)$ \pause definida por
  \begin{equation*}
    \omega(Y)=
    \begin{cases}
      \xi & \text{si } Y=\sigma(\xi), \\
      0 & \text{si } Y \text{ es horizontal.}
    \end{cases}
  \end{equation*}
  \pause

  Propiedades:\pause
  \begin{itemize}
    \item $H=\ker \omega$ \pause
    \item $R^*_g\omega=\ad_{g^{-1}}\circ \omega$.
  \end{itemize}
\end{frame}

\begin{frame}{Conexiones como campos gauge}
  $\mathcal{U}$ recubrimiento de $B$ por abiertos trivializantes y $\left\{ s_U:U\rightarrow p^{-1}(U):U\in \mathcal{U} \right\}$ familia de secciones locales. El \emph{campo gauge} asociado a una $1$-forma de conexión $\omega\in \Omega^1(P;\GG)$ es
  \begin{equation*}
    \left\{ A_U=s^*_U\omega \in \Omega^1(U;\GG): U\in \mathcal{U} \right\}.
  \end{equation*}
  \pause

  Se cumple:
\begin{equation*}
  \omega|_{p^{-1}(U)}=\ad_{g^{-1}_U}\circ p^*A_U + g^*_U \theta,
\end{equation*}
con $\theta\in \Omega^1(G; \GG)$ la \emph{$1$-forma de Maurer-Cartan}, definida por $\theta_g=(L_{g^{-1}})_*$.

\pause 
Además,
\begin{equation*}
  A_U=\ad_{g_{UV}}\circ A_V + g_{VU}^* \theta.
\end{equation*}
\end{frame}

\begin{frame}{Curvatura}
  $p:P\rightarrow B$ fibrado principal, $\omega$ $1$-forma de conexión. \pause
  
  Cualquier vector $Y_y \in T_yP$ se descompone como $Y_y=Y_y^v+Y_y^h$.
  \pause

  La \emph{curvatura} de la conexión definida por $\omega$ es la $2$-forma $\Omega\in \Omega^2(P;\GG)$ definida por
  \begin{equation*}
    \Omega(Y_y,Z_y)=d\omega(Y_y^h,Z_y^h).
  \end{equation*}
  \pause

  Interpretación geométrica:
  \begin{equation*}
    \Omega(Y,Z)=Y^h\omega(Z^h)-Z^h\omega(Y^h)-\omega([Y^h,Z^h])=-\omega([Y^h,Z^h]),
  \end{equation*}
  luego $\Omega$ se anula $\Leftrightarrow$ $[Y^h,Z^h]$ es horizontal.
  \pause

  Propiedades:
  \begin{itemize}
    \item $\Omega=d\omega+[\omega,\omega]$ (Ecuación de estructura)
    \item $d\Omega(Y^h,Z^h,W^h)=0$ (Identidad de Bianchi)
  \end{itemize}
\end{frame}

\begin{frame}{Curvatura como fuerza de campo gauge}
  $\mathcal{U}$ recubrimiento de $B$ por abiertos trivializantes y $\left\{ s_U:U\rightarrow p^{-1}(U):U\in \mathcal{U} \right\}$ familia de secciones locales. La \emph{fuerza de campo gauge} asociada a la curvatura $\Omega\in \Omega^2(P;\GG)$ de una $1$-forma de conexión $\omega\in \Omega^1(P;\GG)$ es
  \begin{equation*}
    \left\{ F_U=s^*_U\Omega \in \Omega^2(U;\GG): U\in \mathcal{U} \right\}.
  \end{equation*}
  \pause
  Propiedades:
  \begin{itemize}
    \item $F_U=dA_U+[A_U,A_U]$ (de la ecuación de estructura)
      \pause
    \item $F_U=\ad_{g_{UV}}\circ F_V$.
  \end{itemize}
\end{frame}

\begin{frame}{Fibrados asociados}
  $p:P\rightarrow B$ fibrado principal con fibra $G$ que actúa sobre $F$ por la izquierda. \pause

  Se define el \emph{fibrado asociado} $P\times_G F$ como el cociente $(P\times F)/G$ por la acción
  \begin{align*}
     (P\times F)\times G&\longrightarrow P\times F\\ 
     ((y,f),g) &\longmapsto (y\cdot g,g^{-1}\cdot f),
    \end{align*}
    \pause
    con la aplicación
    \begin{align*}
      p_F :P\times_G F&\longrightarrow B\\ 
      [(y,f)] &\longmapsto p(y). 
      \end{align*}
\end{frame}

\begin{frame}{Ejemplos}
  \begin{itemize}
    \item $\SF^1\rightarrow \SF^1$. Se obtiene la banda de Moebius con la acción $(f,\pm 1)\mapsto \pm f$ y el cilindro con la acción trivial.
      \pause
    \item $M$ variedad diferenciable y $L(M)\rightarrow M$ el fibrado de referencias. Hay un isomorfismo
      \begin{align*}
	L(M)\times_{\text{GL}(n,\RR)}\RR^n&\longrightarrow TM\\ 
	[(\psi,\mathbf{v})] &\longmapsto \psi(\mathbf{v}). 
	\end{align*}
	\pause
      \item $E \rightsquigarrow P(E)$, $E\cong P(E)\times_G F$.
  \end{itemize}
\end{frame}

\begin{frame}{Fibrado adjunto y curvatura}
  \begin{itemize}
    \item Se llama \emph{fibrado adjunto} de un $G$-fibrado principal $P\rightarrow B$ al fibrado $\ad\ P=P\times_G \GG$, donde el cociente se realiza por la acción dada por la representación adjunta $\ad\ :G\rightarrow \text{Aut}(\GG)$.
      \pause
    \item El fibrado adjunto permite ver la curvatura de una conexión como una $2$-forma definida sobre la base: \pause
      Si tenemos una conexión en $P$ con curvatura $\Omega$ podemos definir $\tilde{\Omega}\in \Omega(B,\ad\ P)$ como
      \begin{equation*}
	\tilde{\Omega}_x(X_x,Y_x)=[(y,\Omega_y(X_y^h,Y_y^h))],
      \end{equation*}
      con $X_y^h$, $Y_y^h$ los levantamientos horizontales: $X_y^h\in H_y$ único tal que $p_*(X^h_y)=X_x$, $x=p(y)$.
  \end{itemize}
\end{frame}

\begin{frame}{Clases características}
  Buscamos una generalización del teorema de Gauss-Bonnet:
  \only<1>{
  \begin{equation*}
      \int_M K v_g= 2\pi \chi(M).
  \end{equation*}
}
  
    \only<2>{ \begin{equation*}
      \underbrace{\int_M K v_g}_{\text{Geometría}} = \underbrace{2\pi \chi(M)}_{\text{Topología}}.
    \end{equation*}}
  \end{frame}
  \begin{frame}{Clases características}
  
       En primer lugar buscamos algo que se pueda integrar en toda la variedad:
       \pause
       \begin{enumerate}
	 \item Fijamos un fibrado principal $P\rightarrow M$ y una conexión en $P$. Consideramos la curvatura $\tilde{\Omega} \in \Omega^2(M; \ad\ P)$. 
	   \pause
	 \item Si $n=2k$ podemos tomar $\Omega \wedge \overset{(k)}{\cdots} \wedge \Omega$, que se podría integrar en $M$ si tuviera valores reales.
	   \pause
	 \item Por tanto, vamos a buscar funciones $f:\ad\ P\rightarrow \RR$.
       \end{enumerate}
\end{frame}

\begin{frame}{Clases características}
  \begin{itemize}
    \item $S^k(\GG)=\left\{ f:\GG\times\overset{(k)}{\cdots} \times \GG \rightarrow \RR \text{ multilineales y simétricas.}\right\}$
      \pause
    \item Llamo $\RR[\GG]_k$, \emph{polinomios homogéneos} a funciones $F:\GG\rightarrow \RR$ tales que, fijado un isomorfismo $\GG\rightarrow \RR^m$, el polinomio $F(\mathtt{x}^1,\dots,\mathtt{x}^m)$ dado por
	  \begin{center}
	\begin{tikzcd}[ampersand replacement=\&]
	  \GG \arrow{r}{F}\arrow{dr} \& \RR    \\ 
	  \& \RR^m \arrow{u}[anchor=west]{F(\mathtt{x}^1,\dots,\mathtt{x}^m)}
	  \end{tikzcd}
	\end{center}
	es homogéneo. \pause Hay una biyección $S^k(\GG) \leftrightarrow \RR[\GG]_k$.
	\pause
      \item Llamo $I^k(\GG)\subset S^k(\GG)$, \emph{polinomios invariantes} a las $f\in S^k(\GG)$ tales que
	\begin{equation*}
	  f(\ad_g \xi_1,\dots,\ad_g \xi_k)=f(\xi_1,\dots,\xi_k).
	\end{equation*}
  \end{itemize}
\end{frame}
\begin{frame}{Clases características}
  \begin{theorem}[Construcción de Weil de las clases características]
    Sea $p:P\rightarrow B$ un fibrado principal con fibra $G$, $\omega$ una $1$-forma de conexión en $P$ con curvatura $\Omega$ y $f\in I^k(\GG)$. \pause La $2k$-forma $f(\Omega \wedge \overset{(k)}{\cdots}\wedge \Omega)$ en $P$ definida por
    \begin{align*}
      f(\Omega &\wedge \overset{(k)}{\cdots}\wedge \Omega)(Y_1,\dots,Y_{2k})= \\ & =\frac{1}{2k!}\sum_{\sigma \in \mathfrak{S}_{2k}}(-1)^{\sigma} f(\Omega(Y_{\sigma(1)},Y_{\sigma(2)}),\dots,\Omega(Y_{\sigma(2k-1)},Y_{\sigma(2k)}))
    \end{align*}
    tiene las siguientes propiedades:
    \pause
    \begin{enumerate}
      \item Se puede proyectar (es la pullback por $p$ de una $2k$-forma en $B$).
	\pause
      \item Es cerrada.
	\pause
      \item La clase de cohomología de su proyección en $B$ no depende de la elección de la conexión $\omega$. Esta clase se llama la \emph{clase característica del fibrado $P\rightarrow B$ asociada a $f$}.
    \end{enumerate}
  \end{theorem}
\end{frame}
\begin{frame}{Ejemplo: Fibrados $U(1)$}
 $\SF^2$ y $U_N$, $U_S$ los hemisferios norte y sur. \pause

 Consideramos un fibrado principal $U(1)$ dado por la función de transición
 \only<2>{
	  \begin{center}
	\begin{tikzcd}[ampersand replacement=\&]
	  U_N\cap U_S \arrow{r}{\psi} \& U(1)    
	  \end{tikzcd}
	\end{center}
      }
 \only<3>{
	  \begin{center}
	\begin{tikzcd}[ampersand replacement=\&]
	  U_N\cap U_S\arrow{d}{\sim} \arrow{r}{\psi} \& U(1)\arrow{d}{\sim} \\
	  \SF^1 \arrow{r} \& \SF^1
	  \end{tikzcd}
	\end{center}
      }
 \only<4>{
	  \begin{center}
	\begin{tikzcd}[ampersand replacement=\&]
	  U_N\cap U_S\arrow{dd}{\sim} \arrow{r}{\psi} \& U(1)\arrow{dd}{\sim} \\
	  \\
	  \SF^1 \arrow{r} \& \SF^1 \\
	  \phi\ (\text{mod}\ 2\pi) \arrow[mapsto]{r} \& k\phi\ (\text{mod}\ 2\pi).
	  \end{tikzcd}
	\end{center}
      }
      \pause
      Para cada grado $k\in \ZZ$ tenemos un fibrado principal distinto $P_k\rightarrow \SF^2$.
\end{frame}
\begin{frame}{Ejemplo: Fibrados $U(1)$}
  Consideramos la conexión definida por un campo gauge $A$ de la forma \pause
  \begin{itemize}
    \item en $U_N$, $A_N=0$, \pause
    \item en $U_N\cap U_S$, $A_S=\psi A_N + \psi d\psi = \psi d\psi$ y lo extendemos de cualquier manera a todo $U_S$.
  \end{itemize}
  \pause
  La curvatura es $F=dA+[A,A]=dA$.
  \pause
  Como $U(1)$ es abeliano, todos los polinomios son invariantes y puedo escoger $p(\mathtt{x})=\frac{1}{2\pi}\mathtt{x}$.
  \pause
  Finalmente,
	       \begin{align*}
		 \int_{\SF^2} p(\Omega)&=\frac{1}{2\pi}\int_{U_S}dA_S=\frac{1}{2\pi}\int_{U_N\cap U_S}A_S|_{U_N\cap U_S}\\ &=\frac{1}{2\pi}\int_{U_N\cap U_S}\psi d\psi= \frac{1}{2\pi} \int_0^{2\pi} k\phi d\phi = k.
	       \end{align*}
\end{frame}
 \begin{frame}{Referencias}
   \nocite{*}
   \bibliographystyle{plain}
   \bibliography{biblio2}
 \end{frame}

\end{document}
